\section{�rvores}
\label{arvores}

\newcommand{\Tree}{{\it Tree\/}}
\newcommand{\Leaf}{{\it Leaf\/}}
\newcommand{\Node}{{\it Node\/}}
\newcommand{\BTree}{{\it BTree\/}}
\newcommand{\ArvoreBinariaDeInteiros}{{\it �rvoreBin�riaDeInteiros\/}}
\newcommand{\ArvoreDeInteiros}{{\it �rvoreDeInteiros\/}}
\newcommand{\primogenito}{{\it primog�nito\/}}
\newcommand{\irmao}{{\it irm�o\/}}
\newcommand{\pai}{{\it pai\/}}
\newcommand{\esq}{{\it esq\/}}
\newcommand{\dir}{{\it dir\/}}

A estrutura de dados b�sica pode ser vista como uma estrutura de dados
recursiva, definida como i) vazia (ou uma folha) ou ii) um nodo
contendo um elemento e um certo n�mero de ramos (ou nodos), que cont�m
sub-�rvores:

  \[ \text{{\tt data \Tree\ $a$ = \Leaf\ | \Node\ $a$ [\Tree\ $a$]}} \]

Os elementos podem tamb�m estar nas folhas, em vez de nos nodos
internos:

  \[ \text{{\tt data \Tree'\ $a$ = \Leaf\ $a$ | \Node\ [\Tree\ $a$]}} \]

Se a �rvore tem duas sub-�rvores (possivelmente vazias), ela � chamada
de �rvore bin�ria:

  \[ \text{{\tt data \BTree\ $a$ = \Leaf\ $a$ | \Node\ (\Tree\ $a$) (\Tree\ $a$)}} \]

Em computa��o, uma �rvore pode ser vista tamb�m como um grafo ---
conjunto de v�rtices e arestas entre esses v�rtices --- sem ciclos e
conexo (se n�o for conexto, seria uma floresta). No entanto, n�o vamos
abordar grafos neste livro.

Em linguagens que prov�em suporte ao uso de ponteiros mas n�o �
defini��o e manipula��o direta de tipos recursivos, a representa��o de
�rvores bin�rias pode ser feita com o uso de ponteiro como mostra o
exemplo a seguir:

  \progb{
        \struct\ \ArvoreBinariaDeInteiros\ \{ \\
        \hspace*{.2cm} int \eelem; \\
        \hspace*{.2cm} \struct\ \ArvoreBinariaDeInteiros\ * $\esq,\dir$; \\
        \}
  }
Os campos \esq\ e \dir\ de um nodo s�o ponteiros, possivelmente nulos,
para sub-�rvores.

Para �rvores n�o bin�rias, pode ser usada uma representa��o, que
podemos chamar de {\em representa��o com
  primog�nito-irm�o-e-pai\/}, como a seguir:

  \progb{
        \struct\ \ArvoreDeInteiros\ \{ \\
        \hspace*{.2cm} int \eelem; \\
        \hspace*{.2cm} \struct\ \ArvoreDeInteiros\ *\primogenito, \irmao, \pai; \\ \} }

A �rvore da Figura \ref{Arv1} � mostrada na Figura \ref{Rep-arv1}.

\begin{figure}

....

\label{Arv1}
\end{figure}

\begin{figure}

....

\label{Rep-arv1}
\end{figure}
