% !TEX encoding = ISO-8859-1
\chapter{Algoritmos e Árvores de Pesquisa}
\label{algoritmos-de-pesquisa}

Pesquisar em computação significa encontrar um dado valor, chamado de
{\em chave da pesquisa\/}, dentre vários valores existentes. Os
valores existentes podem estar representados de várias formas, mas
vamos tratar neste livro apenas de listas e árvores.  Mesmo nos
restringindo apenas a essas formas de representação de valores,
existem vários algoritmos de pesquisa. 

Na seção \ref{sec:pesquisa-em-lista} apresentamos um algoritmo simples
de {\em pesquisa sequencial\/} em listas (incluindo representação com
arranjos). Duas variações simples dessa pesquisa sequencial são
apresentadas nos exercícios resolvidos. A primeira é baseada em
pesquisa em lista ordenada, que termina a pesquisa sequencial quando a
chave da pesquisa é encontrada ou quando se torna maior do que um
elemento da lista (supondo ordem crescente dos valores na lista). A
segunda usa o que é chamado de {\em sentinela} --- um elemento
adicionado ao extremo (tipicamente, de arranjo), para evitar teste
para verificar chegada a esse extremo (por isso, é usada somente
quando o número de elementos que pode ser armazenado é limitado, como
ocorre no caso de arranjos).

A seção \ref{sec:pesquisa-binaria} apresenta o eficiente algoritmo de
pesquisa em um arranjo ordenado chamada de {\em pesquisa binária}.  A
seção \ref{sec:arvore-binaria-de-pesquisa} apresenta algoritmos de
pesquisa em árvore binária, incluindo variações do algoritmo básico
(em árvores não balanceadas) para diferentes formas de obter
balanceamento da árvore na qual a pesquisa é feita, com o objetivo de
aumentar a eficiência da pesquisa.

% !TEX encoding = ISO-8859-1
\subsection{Pesquisa binária}
\label{pesquisa-binaria}

Pesquisa binária é apresentada a seguir usando árvores de pesquisa
binária, em Haskell, e arranjos ordenados, em pseudo-código
imperativo.

Uma árvore de pesquisa binária é uma árvore binária --- isto é, um
árvore com duas sub-árvores (possivelmente vazias), digamos, à
esquerda e à direita --- com os elementos contidos nos nodos e tal
que: todo elemento contido em um nodo é maior que os elementos
contidos na sub-árvore à esquerda e menor que os elementos contidos na
sub-árvore à direita.

Dois exemplos de árvores binárias de pesquisa com os elementos de 1 a
7 são mostradas abaixo. 

  4                    4
 / \                  / \
1   5                2   6
 \   \              / \  /\
 3    6            1  3 5  7 
/      \ 
2       7

A propriedade fundamental de uma árvore binária de pesquisa é o acesso
eficiente (como vamos ver, logaritmo) a um elemento.

\subsubsection{Versão funcional}






% !TEX encoding = ISO-8859-1
\section{�rvore bin�ria de pesquisa}
\label{sec:arvore-binaria-de-pesquisa}

Uma �rvore bin�ria de pesquisa � uma �rvore bin�ria --- isto �, um
�rvore com duas sub-�rvores (possivelmente vazias), digamos, �
esquerda e � direita --- com os elementos contidos nos nodos, um
elemento por nodo, e tal que: todo elemento contido em um nodo � maior
que os elementos contidos na sub-�rvore � esquerda deste nodo e menor
que os elementos contidos na sub-�rvore � direita do nodo.

Dois exemplos de �rvores bin�rias de pesquisa com os elementos de 1 a
7 s�o mostradas abaixo. 

\begin{verbatim}
      4                    4
     / \                  / \
    1   5                2   6
     \   \              / \  /\
     3    6            1  3 5  7 
    /      \ 
    2       7
\end{verbatim}

A propriedade fundamental de uma �rvore bin�ria de pesquisa � o acesso
eficiente a um elemento (veja abaixo coment�rio sobre a complexidade
logar�tmica da pesquisa por um elemento no caso m�dio).

Uma �rvore bin�ria de pesquisa � uma importante estruturas de dados,
gerando implementa��es simples de pesquisa, inser��o e remo��o. Em
computa��o, um {\em dicion�rio\/} � um tipo abstrato que define tais
opera��es. As se��es seguintes apresentam vers�es recursivas e
iterativas de pesquisa, inser��o e remo��o, usando �rvores bin�rias de
pesquisa.

\subsection{Vers�o funcional}

Considere a defini��o de �rvore bin�ria com elementos nos nodos
internos da �rvore apresentada na se��o \ref{sec:arvores}:

\begin{center}
\begin{tabular}{l}
\begin{hask}{ArvB,FolhaB,NodoB}{White}
 data ArvB a = FolhaB | NodoB a (ArvB a) (ArvB a)
\end{hask}
\end{tabular}
\end{center}

\subsubsection{Pesquisa}

A pesquisa por um elemento em uma �rbore de pesquisa bin�ria � um
algoritmo simples, mostrado a seguir:

\begin{center}
\begin{tabular}{l}
\begin{hask}{pesq}{\decremento}
pesq v FolhaB          = False
pesq v (NodoB v' t t') = case compare v v' of
                           LT -> pesq v t
                           GT -> pesq v t'
                           _  -> True
\end{hask}
\end{tabular}
\end{center}

A complexidade de tempo de \inh{pesq} � determinada a seguir: 

\begin{enumerate}

  \item Vamos considerar que o tamanho da entrada � dado pelo n�mero
    de elementos $n$ na �rvore.

  \item A opera��es relevantes s�o a compara��o (uso de \inh{compare})
    e o casamento de padr�o.

  \item \label{complexidade-da-pesquisa-binaria-no-pior-caso} A
    express�o-determinante-da-efici�ncia �, no pior caso, igual a
    $T(n-1) + k$, onde $k$ � uma constante que expressa o tempo de
    execu��o da compara��o. No pior caso, o elemento est� em
    sub-�rvore que cont�m $n-1$ nodos: todos os nodos menos o nodo
    corrente. Neste caso, a �rvore est� desbalanceada. Isso ocorre,
    por exemplo, para as �rvores mostradas a seguir, constru�das com
    inser��es sucessivas de elementos em uma lista crescente e
    decrescente, respectivamente.

    \begin{verbatim}
         v0                             v0
          \                             /
          v1                           v1
            \                         /
            ...                     ...
              \                     /
              vn-1               vn-1
                \                 /
                 vn              vn
    \end{verbatim}

  \end{enumerate}

Assim, a rela��o de recorr�ncia neste caso �:

   \[ T(n) = T(n-1) + k \]

A solu��o dessa rela��o de recorr�ncia foi apresentada na se��o
\ref{sec:maior-elemento}:

       \[ T(n) \asymp n \]

Note que o algoritmo de pesquisa em uma �rvore bin�ria de pesquisa tem
complexidade $O(p)$ no pior caso, onde $p$ � a profundidade da �rvore
(n�mero de arestas do maior caminho existente entre a raiz e uma
folha). 

Em uma �rvore balanceada, o n�mero de elementos $n$ � igual a $2^p$,
ou seja, em uma �rvore balanceada, a pesquisa tem complexidade 

  \[ T(n) \asymp lg n \]
Em uma �rvore totalmente desbalanceada, temos no entanto $p = n-1$.

\subsubsection{Inser��o}

O algoritmo para inser��o de elementos em uma �rvore bin�ria de
pesquisa � semelhante ao algoritmo de pesquisa:

\begin{center}
\begin{tabular}{l}
\begin{hask}{ins}{\decremento}
ins v FolhaB             = NodoB v FolhaB FolhaB
ins v t@(NodoB v' t1 t2) = case compare v v' of
                             LT -> ins v t1
                             GT -> ins v t2
                             _  -> t
\end{hask}
\end{tabular}
\end{center}

O tempo de complexidade do algoritmo de inser��o � o mesmo do de
pesquisa: $lg n$ no caso de uma �rvore balanceada, mas linear no pior
caso de uma �rvore desbalanceada.

\subsubsection{Remo��o}

Para remo��o de um nodo $n$, raiz de uma �rvore $t$, o algoritmo de
remo��o de elemento essencialmente subsititui $n$ pelo menor elemento
da sub-�rvore direita de $t$ (e remove este menor elemento dessa
sub-�rvore). O algoritmo em Haskell � apresentado a seguir.

\begin{center}
\begin{tabular}{l}
\begin{hask}{remov}{\decremento}
remov:: Ord a => a -> ArvB a -> ArvB a
remov _ FolhaB = FolhaB
remov a (NodoB b l r) = case compare a b of 
                          LT -> NodoB b (remov a l) r
                          GT -> NodoB b l (remov a r)
                          _  -> junta l r

junta:: Ord a => ArvB a -> ArvB a -> ArvB a
-- junta l r cria �rvore com o menor elemento m de r como raiz e remove m de r, 
-- se m existir; sen�o (r � vazia) retorna l.
junta l r = case min r of 
              Nothing     -> l
              Just (m,r') -> Node m l r'

min:: Ord a => ArvB a -> Maybe (a,ArvB a)
-- min t retorna Just (m,r) se o menor elemento m de t existir
-- (onde r � a sub-�rvore direita de t); sen�o Nothing.
min FolhaB             = Nothing
min (NodoB a FolhaB r) = Just (a, r)
min (NodoB a l      r) = Just (m,r)
  where Just (m,_)     = min l
\end{hask}
\end{tabular}
\end{center}

O tempo de complexidade do algoritmo de remo��o de elementos � tamb�m
o mesmo do de pesquisa: $lg n$ no caso de uma �rvore balanceada, mas
linear no pior caso de uma �rvore desbalanceada.

\subsection{Vers�o imperativa}

Vamos usar a defini��o de �rvore bin�ria apresentada na se��o
\ref{sec:arvores}:

\begin{center}
\begin{tabular}{l}
\begin{alg}{ArvoreBinariaDeInteiros}{White}
struct ArvoreBinariaDeInteiros
     int elem
     struct ArvoreBinariaDeInteiros *esq, dir 
\end{alg}
\end{tabular}
\end{center}

\subsubsection{Pesquisa}

A pesquisa iterativa por um valor em uma �rvore bin�ria de pesquisa
com valores inteiros � apresentada a seguir:

\begin{center}
\begin{tabular}{l}
\begin{alg}{pesq}{\decremento}
pesq(k, arvBin) 
    nodoCorrente = arvBin
    while nocoCorrente != NULL
        v = nocoCorrente->elem
        if v > k 
           nodoCorrente = nodoCorrente->esq else
        if v < k
            nodoCorrente = nodoCorrente->dir
        else return nodoCorrente
    return NULL
\end{alg}
\end{tabular}
\end{center}

A complexidade � a mesma da vers�o recursiva: $lg n$ no caso de uma
�rvore balanceada, mas linear no pior caso de uma �rvore
desbalanceada.

\subsubsection{Inser��o}

A inser��o de elemento na vers�o imperativa sup�e que o elemento a ser
inserido � um novo elemento (n�o ocorre na �rvore). � usado comando de
repeti��o em vez de recurs�o, e comando de atribui��o � usado para
modificar a �rvore na qual o elemento est� sendo inserido, em vez de
criar nova �rvore.

Um novo nodo � criado com chamada a \ina{novoNodo}, que atribui
refer�ncias nulas aos campos \ina{esq} e \ina{dir} e atribui o
argumento ao campo \ina{elem}.

\begin{center}
\begin{tabular}{l}
\begin{alg}{ins}{\decremento}
ins (int k, ArvoreBinariaDeInteiros* arvBin) {
  ArvoreBinariaDeInteiros* nodoCorrente = arvBin
  ArvoreBinariaDeInteiros* prev = NULL
  int v
  while (nodoCorrente != NULL) 
    v = nodoCorrente->elem;
    prev = nodoCorrente;
    if (k < v) 
       nodoCorrente = nodoCorrente -> esq
    else 
       nodoCorrente = nodoCorrente -> dir

  if prev == NULL
    arvBin = novoNodo(k)
  else 
    if (k < prev->elem) 
      prev->esq = novoNodo(k)
    else 
      prev->dir = novoNodo(k)
\end{alg}
\end{tabular}
\end{center}

A complexidade de \ina{ins} � a mesma da vers�o recursiva: $lg n$ no
caso de uma �rvore balanceada, mas linear no pior caso de uma �rvore
desbalanceada.

A vers�o imperativa do algoritmo de remo��o de um elemento em uma
�rvore bin�ria de pesquisa � deixada como exerc�cio para o leitor
(exerc�cio
\ref{ex:remocao-de-elemento-em-arvore-binaria-de-pesquisa}).

\HRule

%Embora a complexidade da pesquisa a um elemento seja logar�tmica no
%caso m�dio, como vimos no item
%\ref{complexidade-da-pesquisa-binaria-no-pior-caso} acima, no pior
%caso a complexidade da pesquisa � a mesma da complexidade da pesquisa
%sequencial, quando a �rvore est� totalmente desbalanceada, ficando
%equivalente a uma lista.

H� muitos trabalhos de pesquisa em computa��o que procuram preservar o
balanceamento de �rvores bin�rias de pesquisa, para manter a
propriedade de complexidade logar�tmica para as opera��es de inser��o,
remo��o, pesquisa e ordena��o de valores, baseados principalmente em
t�cnicas de transforma��o: simplifica��o ou mudan�a de representa��o.

A simplifica��o transforma �rvores n�o balanceadas em �rvores
balanceadas (segundo alguma crit�rio de balanceamento) sem adicionar
nenhuma informa��o adicional a nodos da �rvore. A se��o
\ref{sec:arvore-AVL} aborda {\em �rvores AVL\/}, nas quais a diferen�a
entre a altura das sub-�rvores de qualquer nodo n�o deve ser maior que
1. A altura de um nodo da �rvore � o n�mero de nodos do maior caminho
entre esse nodo e uma folha. A altura de uma �rvore � altura do nodo
raiz da �rvore.

Outro estrutura de dados que prov� balanceamento de �rvores de
pesquisa baseada na t�cnica de transforma��o por simplifica��o � a que
vamos chamar de {\em �rvores MovPraRaiz\/} (em ingl�s, {\em splay
  trees\/}), baseada na ideia de mover para raiz um elemento em
opera��es de inser��o, remo��o ou pesquisa.  �rvores MovPraRaiz s�o
abordadas no exerc�cio resolvido \ref{ex:arvores-MovPraRaiz}.

Exemplos de balanceamento via mudan�a de representa��o ocorrem com os
seguintes tipos de �rvore: {\em �rvore bicolor\/} (tamb�m chamadas de
vermelha-e-preta ou rubro-negra, em ingl�s {\em red-black tree\/}) e
{\em �rvore B\/} (em ingl�s, {\em B-tree\/}), abordadas
respectivamente nas se��es \ref{sec:arvore-bicolor} e
\ref{sec:arvore-B}.


% !TEX encoding = ISO-8859-1
\section{�rvore AVL}
\label{sec:arvore-AVL}

O nome AVL � proveniente das iniciais dos sobrenomes dos dois
pesquisadores russos G.~M.~\underline{A}delson-\underline{V}elsky e
E.~M.~\underline{L}andis, que foram os primeiros a definir e realizar
trabalhos com esse tipo de �rvore.

Seja $n$ um n� de uma �rvore bin�ria, $ad_n$ e $ae_n$ as alturas da
sub-�rvore esquerda e direita de $n$, respectivamente, e seja $k_n =
ae_n - ad_n $ o {\em fator de balanceamento\/} do nodo (i.e.~o fator
de balanceamento � igual ao valor da diferen�a entre as alturas de
suas sub-�rvores).

Uma �rvore AVL � uma �rvore de pesquisa bin�ria na qual $\delta_n$ �
igual a 0 ou 1 ou -1, para todo nodo $n$.

Por exemplo, a �vore bin�ria de pesquisa abaixo � esquerda � uma
�rvore AVL, enquanto a da direita n�o �. 

\begin{verbatim}
      5                 5
     / \               /
    2   6             2 
   /                 / 
  1                 1   
\end{verbatim}

O algoritmo de pesquisa em uma �rvore AVL � o mesmo do algoritmo de
pesquisa em uma �rvore bin�ria de pesquisa. 

Os algoritmos de inser��o e remo��o s�o apresentados nas subse��es
seguintes.

\subsection{Inser��o}
\label{sec:insercao-em-arvores-AVL-versao-func}

Ap�s inser��o de nodo em uma �rvore AVL, � feita uma verifica��o do
fator de balanceamento de cada nodo que est� no caminho da raiz at� o
nodo inserido. Se a inser��o tornar o fator de balanceamento maior que
1 ou menor que -1, a sub-�rvore com raiz nesse nodo � rebalanceada,
por meio de uma {\em rota��o}, para que a condi��o-AVL volte a ser
satisfeita. Como a inser��o de elemento em uma �rvore pode aumentar a
altura da �rvore em no m�ximo 1, o fator de balanceamento deve ser,
logo ap�s a inser��o de nodo na sub-�rvore esquerda e antes do
rebalanceamento, no m�ximo igual a 2; e, logo ap�s a inser��o de nodo
na sub-�rvore direita e antes do rebalanceamento, no m�nimo igual a
-2.

Quando o fator de balanceamento � igual a 2, existem duas
possibilidades (outras duas possibilidades, que existem quando o fator
de balanceamento � igual a -2, s�o an�logas). 

\newcommand{\altura}{{\it altura\/}}

A primeira, mostrada no caso \verb+sobE+ abaixo, temos 
  $\text{\altura\ (\verb+ee+) = \altura\ (\verb+d+)}$.
Note que: 
  \begin{enumerate}
    \item se
      $\altura\ (\verb+ee+) < \altura\ (\verb+d+)$ ent�o a
      �rvore continuaria sendo uma �rvore AVL ap�s a inser��o de um
      nodo em \verb+ee+;
    \item se $\altura\ (\verb+ee+) > \altura\ (\verb+d+)$ ent�o a
      �rvore j� n�o seria uma �rvore AVL antes da inser��o de um nodo
      em \verb+ee+.
  \end{enumerate}
   
\begin{verbatim}
       Caso sobE      Caso sobED
           v              v
         /   \          /   \   
        ve    d        ve    d
       /  \           /  \                    
      ee  ed         ee   ved
                         / \
                       ede  edd                   

�rvore depois de rotacionada:

    Caso sobE          Caso sobED
     ve                   ved
    /  \                /     \ 
   ee   v             ve       v
       / \           /  \     / \ 
      ed  d         ee  ede edd  d

\end{verbatim}

O caso \verb+sobED+ pode ser expresso como \verb+sobE+ (aplicado ao
nodo com raiz \verb+ved+) seguido de \verb+sobD+ (aplicado ao mesmo
nodo).

%� importante notar que apenas um fator de balanceamento n�o �
%suficiente para determinar se uma �rvore AVL necessita de rota��o ap�s
%uma inser��o. Por exemplo, considere as duas �rvores AVL a seguir:
%
%\begin{verbatim}
%      7                  7
%     /  \               /
%    3    8             3 
%   / \    \ 
%  2   5    9
% /     \ 
%1       6
%\end{verbatim}
%
%O fator de balanceamento da �rvore com raiz \verb+7+ � igual a 1, nas
%duas �rvores acima, antes da inser��o.  No entanto, a inser��o de
%\verb+4+ n�o quebra a condi��o de a �rvore � esquerda ser AVL, ao
%contr�rio do que ocorre no caso da �rvore � direita; ap�s a inser��o,
%e antes da rota��o, que deve ser feita apenas na �rvore � direita,
%temos:
%
%\begin{verbatim}
%     AVL              N�o AVL 
%      7                  7
%     / \                /
%    3   8              3 
%   / \   \             \ 
%  2   5   9             4
% /   / \ 
%1   4   6
%\end{verbatim}

A tarefa de determinar, usando apenas o pr�prio fator de
balanceamento, a varia��o do fator de balanceamento ap�s uma inser��o,
e demonstrar que tal varia��o � verificada em todos os casos, �
deixada para trabalho futuro. N�o encontramos na literatura textos que
abordam o assunto de forma clara e precisa.

O armazenamento da altura em cada nodo evita ter que calcular a altura
de cada nodo que est� no caminho da �rvore at� o nodo inserido, o que
seria desnecessariamente ineficiente. 

\subsubsection{Vers�o funcional}

A inser��o de elemento em �rvore AVL � feita em Haskell como a seguir:

\begin{center}
\begin{tabular}{l}
\begin{hask}{ins}{\decremento}
module AVL (ArvoreAVL, arvVazia, ins) where

type Altura      = Integer
data ArvoreAVL a = Vazia | Nodo a Altura (ArvoreAVL a) (ArvoreAVL a)

arvVazia = Vazia

ins:: (Show a, Ord a) => a -> ArvoreAVL a -> ArvoreAVL a
ins k Vazia              = Nodo k 1 Vazia Vazia
ins k arv@(Nodo v _ e d) = 
 case compare k v of 
  LT -> let e1@(Nodo v1 a1 _ _) = ins k e 
            ad = altura d
         in if a1 - ad == 2 -- condi��o AVL precisa ser restaurada
            then if k < v1  
                 then sobE  (Nodo v undefined e1 d)  
                 else sobED (Nodo v undefined e1 d)
            else Nodo v (max a1 ad + 1) e1 d
  GT -> let d1@(Nodo v1 a1 _ _) = ins k d 
            ae = altura e
         in if a1 - ae == 2 -- condi��o AVL precisa ser restaurada 
            then 
               if k > v1
               then sobD  (Nodo v undefined e d1) 
               else sobDE (Nodo v undefined e d1) 
            else Nodo v (max ae a1 + 1) e d1
  _ -> arv

sobE :: ArvoreAVL a -> ArvoreAVL a
sobE (Nodo v _ (Nodo ve _ ee ed) d) = Nodo ve a ee (Nodo v ad ed d) 
  where ad = max (altura ed) (altura d) + 1
        a  = max (altura ee) ad         + 1

sobD :: ArvoreAVL a -> ArvoreAVL a
sobD (Nodo v _ e (Nodo vd _ de dd)) = Nodo vd a (Nodo v ae e de) dd
  where ae = max (altura e) (altura de) + 1
        a  = max ae         (altura dd) + 1

sobED :: ArvoreAVL a -> ArvoreAVL a
sobED (Nodo v _ (Nodo ve _ ee (Nodo ved _ ede edd)) d) = 
       Nodo ved a (Nodo ve ae ee ede) (Nodo v ad edd d)
  where a  = max ae           ad           + 1
        ae = max (altura ee ) (altura ede) + 1
        ad = max (altura edd) (altura d  ) + 1

sobDE :: ArvoreAVL a -> ArvoreAVL a
sobDE (Nodo v _ e (Nodo vd _ (Nodo vde _ dee ded) dd)) = 
       Nodo vde a (Nodo v ae e dee) (Nodo vd ad ded dd)
  where a  = max ae           ad           + 1
        ad = max (altura ded) (altura dd ) + 1
        ae = max (altura e  ) (altura dee) + 1 

altura (Nodo _ a _ _) = a
altura Vazia          = 0
\end{hask}
\end{tabular}
\end{center}

No pior caso, temos $T(h) = T(h-1) + k$, onde $h$ � a altura da �rvore
e $k$ � o tempo de execu��o referente aos c�lculos de i) altura de uma
sub-�rvore, ii) condi��o AVL e iii) rota��o (uma das fun��es
\inh{sobE}, \inh{sobED}, \inh{sobD}, \inh{sobDE}). Todos os tempos de
i) a iii) t�m complexidade $O(1)$. Logo (cf.~se��o
\ref{sec:maior-elemento}): $T(h) \asymp h$, ou seja, considerando que
$h \asymp lg n$ em uma �rvore balanceada (onde $n$ � o n�mero de
elementos da �rvore):

  \[ T(n) \asymp lg n \]

\subsubsection{Vers�o imperativa}

A vers�o imperativa, mostrada abaixo, usa:
  
  \begin{itemize}

    \item fun��o \ina{novoNodo} e fun��o \ina{malloc}, como em \C:
      \ina{malloc} i) aloca �rea de mem�ria para conter registro
      \ina{AVL} --- o tamanho da �rea a ser alocada � passada para
      \ina{malloc}, sendo o tamanho calculado pela fun��o \ina{sizeof}
      ---, e ii) retorna apontador para �rea alocada;

    \item express�es condicionais, como em \C\ (introduzida na se��o
      \ref{pesquisa-sequencial-em-arranjo-versao-imp});

    \item em \C, � necess�rio definir: \ina{typedef struct AVL AVL;}
      para poder usar apenas \ina{AVL} em vez de \ina{struct AVL};
      al�m disso, o uso de \ina{struct AVL} em vez de apenas \ina{AVL}
      em campo de \ina{AVL} � devido a como � definido (requerido) na
      linguagem \C.

 \end{itemize}

\begin{center}
\begin{tabular}{l}
\begin{alg}{ins}{\decremento}
struct AVL 
  int chave, altura
  struct AVL *esq, *dir

int altura (AVL *p)
  return (p == NULL ? 0 : p->altura)
 
int max (int a, int b) 
  return (a > b ? a : b)
 
AVL* novoNodo (int chave)
    AVL* nodo    = (AVL*) malloc(sizeof(AVL))
    nodo->chave  = chave
    nodo->esq    = nodo->dir = NULL
    nodo->altura = 1
    return nodo

AVL* sobE (AVL* v)
    AVL* ve    = v ->esq
    AVL* ved   = ve->dir
    ve->dir    = v
    v ->esq    = ved
    v ->altura = max (altura (v->esq ), altura (v->dir ))+1
    ve->altura = max (altura (ve->esq), v->altura       )+1
    return ve
 
AVL* sobD (AVL* v) 
    AVL* vd    = v ->dir
    AVL* vde   = vd->esq
    vd->esq    = v
    v ->dir    = vde
    v ->altura = max (altura (v->esq ), altura (v->dir ))+1
    vd->altura = max (v->altura       , altura (vd->dir))+1
    return vd

AVL* sobED (AVL* v) 
  v->esq = sobD (v->esq)  // v->esq->dir sobe (se torna v->esq)
  return sobE(v)          // v->esq      sobe (se torna v)

AVL* sobDE (AVL* v)
  v->dir = sobE (v->dir)  // v->dir->esq sobe (se torna v->dir)
  return sobD(v)          // v->dir      sobe (se torna v)

AVL* ins (AVL* nodo, int chave)
    if (nodo == NULL) return novoNodo (chave)
    if (chave < nodo->chave)
      nodo->esq    = ins (nodo->esq, chave)
      nodo->altura = max (altura (nodo->esq), altura (nodo->dir)) + 1
     if (altura (nodo->esq) - altura (nodo->dir) == 2) // condi��o AVL precisa ser restaurada
	if (chave < nodo->esq->chave) 
	  return sobE (nodo)
	else return sobED (nodo)
    else if (chave > nodo->chave) 
            nodo->dir    = ins (nodo->dir, chave)
            nodo->altura = max (altura (nodo->esq), altura (nodo->dir)) + 1
            if (altura(nodo->dir) - altura(nodo->esq) == 2)  // condi��o AVL precisa ser restaurada
	       if (chave > nodo->dir->chave)
	          return sobD (nodo)
	       else return sobDE (nodo)

    return nodo;
\end{alg}
\end{tabular}
\end{center}

A complexidade de tempo de \inh{ins} � a mesma da vers�o funcional:
$T(n) \asymp lg n$.

\subsection{Remo��o}

\subsubsection{Vers�o funcional}

\subsubsection{Vers�o imperativa}


\section{Árvore Bicolor}
\label{sec:arvore-bicolor}

Uma árvore bicolor --- ou rubro-negra, ou vermelha-e-preta --- é uma
árvore binária de pesquisa com uma informação em cada nodo que indica
sua cor --- vamos chamar um nodo de cor $A$ de nodo $A$ e um nodo de
cor $B$ de nodo $B$ (em geral, as cores $A$ e $B$ são vermelha e
preta) --- e que deve além disso satisfazer às seguintes condições
(chamadas de {\em condições-de-árvore-bicolor\/}):

\begin{enumerate}

\item As folhas e a raiz são nodos $B$.

\item Os filhos de todo nodo $A$ são nodos $B$.

\item Todo caminho de um nodo a uma folha contém o mesmo número de
  nodos $B$.

\end{enumerate}

Um exemplo de uma árvore bicolor é mostrada abaixo.

.... aqui vem figura de árvore vermelha-e-preta. ....

As duas últimas condições garantem a propriedade {\em
  balanceamento-bicolor\/}: o comprimento do caminho mais longo da
raiz a uma folha não é maior que o dobro do comprimento do menor
caminho da raiz a uma folha. Ou seja, a árvore é razoavelmente bem
balanceada em termos de altura dos seus nodos (altura = comprimento do
maior caminho do nodo até uma folha). Como o pior caso da complexidade
de tempo de operações de inserção, remoção e pesquisa são
proporcionais à altura da árvore, isso permite uma eficiência maior do
que árvores binárias de pesquisa.

Para ver porque as duas últimas condições garantem a propriedade de
balanceamento-bicolor, considere que $n_p$ seja o número de nodos $B$
de um caminho a uma folha, para uma árvore $a$.  Seja $k$ o
comprimento do menor caminho $c_m$ da raiz a uma folha (esse caminho
contém $n_b$ nodos pretos). O caminho mais longo da raiz a uma folha
$c_M$ não pode conter nodos vermelhos em níveis consecutivos da árvore
(um novo vermelho não pode ser filho de outro nodo vermelho), e o
número de nodos $B$ tem que ser igual a $n_p$. Portanto, o número de
nodos de $c_M$ tem que ser no máximo igual a $2*k$ (caso em que $c_m$
só tem nodos $B$).

Operações que modificam a árvore geralmente causam reorganização dos
nodos e cores dos nodos de modo a que as condições-de-árvore-bicolor
continuem a ser satisfeitas, o que pode ser feito de modo eficiente.

% The balancing of the tree is not perfect but it is good enough to
% allow it to guarantee searching in O(log n) time, where n is the total
% number of elements in the tree. The insertion and deletion operations,
% along with the tree rearrangement and recoloring, are also performed
% in O(log n) time.

% !TEX encoding = ISO-8859-1
\section{�rvore B}
\label{sec:arvore-B}


\section{Exercícios Resolvidos}

\begin{enumerate}

\item Escreva função que implementa algoritmo de pesquisa em lista
  ordenada.  Ou seja, escreva função que recebe um valor e uma lista,
  que pode-se supor que está ordenada, em ordem não-decrescente, e
  retorne valor booleano que indica se o valor está presente ou não na
  lista. A pesquisa deve considerar, por questão de eficiência, que a
  lista está ordenada.

{\bf Solução}: 

{\bf Versão funcional}:

\begin{center}
\begin{tabular}{l}
\begin{hask}{elemOrdL}{\decremento}
elemOrdL :: Ord a => a -> [a] -> Bool
elemOrdL _ [] = False
elemOrdL a (b:x) 
  | a < b     = elemOrdL a x
  | otherwise = a == b
\end{hask}
\end{tabular}
\end{center}

{\bf Versão imperativa}:

\begin{center}
\begin{tabular}{l}
\begin{alg}{elemOrdL}{\decremento}
elemOrdL (a, l) 
   while ((l != NULL) && (l->elem < a)) l = l->r
   return (l->elem == a)
\end{alg}
\end{tabular}
\end{center}

\item \label{ex:lista-duplamente-encadeada} ....

\item \label{ex:dicionario} Em computação, um dicionário é um tipo
  abstrato que define operações de inserção, remoção e
  pesquisa. Define um tipo abstrato dicionário usando uma árvore
  binária de pesquisa para implementação das operações sobre o tipo
  abstrato.

  {\bf Solução}: ....

\item \label{ex:arvores-MovPraRaiz} Operações em uma árvore MovPraRaiz
  (como inserção, remoção ou pesquisa) podem ser seguidas por uma
  operação básica chamada de {\em movPraRaiz\/} (em inglês, {\em
    splaying\/}), que reorganiza a árvore de modo a mover um elemento
  para a raiz (em operações de inserção e pesquisa, o elemento
  inserido ou pesquisado; em operação de remoção, o elemento a ser
  removido é movido para a raiz antes de ser removido).  A ideia
  básica por trás disso é tornar os elementos usados mais recentemente
  próximos à raiz da árvore e a árvore razoavelmente balanceada.

  A operação \ina{movPraRaiz}, que move um nodo $x$ para a raiz, pode
  realizar um dentre os seguintes passos:
 
  \begin{enumerate}

    \item passo x-raiz: se $x$ é filho da raiz,

    \item passo x-pai-de-x-no-mesmo-lado: 
          $x$ é filho do lado esquerdo e o pai de $x$ está à esquerda da raiz, ou 
          $x$ é filho do lado direito  e o pai de $x$ está à direita da raiz, 

    \item passo x-pai-de-x-em-lados-opostos: 
          $x$ é filho do lado esquerdo e o pai de $x$ está à direita da raiz, ou 
          $x$ é filho do lado direito  e o pai de $x$ está à esquerda da raiz.

  \end{enumerate}

Essas três possibilidades são ilustradas abaixo.

...

\end{enumerate}

\section{Exercícios}

\begin{enumerate}

\item \label{remocao-de-elemento-em-arvore-binaria-de-pesquisa}
  Escreva um programa para remoção de um elemento em uma árvore
  binária de pesquisa que use comandos de repetição em vez de funções
  definidas recursivamente.

\item \label{experimente-alg-ins-arv-AVL} Escreva um programa para
  percorrer e imprimir árvores resultantes de inserções repetidas de
  elementos em uma árvore vazia, usando versão imperativa e/ou
  funcional do algoritmo de inserção em árvore AVL.

\item \label{experimente-alg-rem-arv-AVL} Escreva um programa para
  percorrer e imprimir árvores resultantes de inserções repetidas com
  remoções de alguns elementos inseridos, em uma árvore vazia, usando
  versão imperativa e/ou funcional dos algoritmos de inserção e
  remoção em árvore AVL.

\end{enumerate}
