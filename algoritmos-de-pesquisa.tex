% !TEX encoding = ISO-8859-1
\chapter{Algoritmos de Pesquisa}
\label{algoritmos-de-pesquisa}

Pesquisar em computa��o significa encontrar um dado valor, chamado de
{\em chave da pesquisa\/}, dentre v�rios valores existentes. Os
valores existentes podem estar representados de v�rias formas, mas
vamos tratar neste livro apenas de listas e �rvores.  Mesmo nos
restringindo apenas a essas formas de representa��o de valores,
existem v�rios algoritmos de pesquisa. 

Na se��o \ref{sec:pesquisa-em-lista} apresentamos um algoritmo simples
de {\em pesquisa sequencial\/} em listas (incluindo representa��o com
arranjos). Duas varia��es simples dessa pesquisa sequencial s�o
apresentadas nos exerc�cios resolvidos. A primeira � baseada em
pesquisa em lista ordenada, que termina a pesquisa sequencial quando a
chave da pesquisa � encontrada ou quando se torna maior do que um
elemento da lista (supondo ordem crescente dos valores na lista). A
segunda usa o que � chamado de {\em sentinela} --- um elemento
adicionado ao extremo (tipicamente, de arranjo), para evitar teste
para verificar chegada a esse extremo (por isso, � usada somente
quando o n�mero de elementos que pode ser armazenado � limitado, como
ocorre no caso de arranjos).

A se��o \ref{pesquisa-binaria} apresentamos o eficiente algoritmo de
pesquisa em �rvore bin�ria, chamada de {\em pesquisa bin�ria}. As
se��es seguintes apresentam varia��es da pesquisa bin�ria, que usam
opera��es para balanceamento da �rvore na qual a pesquisa � feita, com
o objetivo de aumentar a efici�ncia da pesquisa.

% !TEX encoding = ISO-8859-1
\subsection{Pesquisa binária}
\label{pesquisa-binaria}

Pesquisa binária é apresentada a seguir usando árvores de pesquisa
binária, em Haskell, e arranjos ordenados, em pseudo-código
imperativo.

Uma árvore de pesquisa binária é uma árvore binária --- isto é, um
árvore com duas sub-árvores (possivelmente vazias), digamos, à
esquerda e à direita --- com os elementos contidos nos nodos e tal
que: todo elemento contido em um nodo é maior que os elementos
contidos na sub-árvore à esquerda e menor que os elementos contidos na
sub-árvore à direita.

Dois exemplos de árvores binárias de pesquisa com os elementos de 1 a
7 são mostradas abaixo. 

  4                    4
 / \                  / \
1   5                2   6
 \   \              / \  /\
 3    6            1  3 5  7 
/      \ 
2       7

A propriedade fundamental de uma árvore binária de pesquisa é o acesso
eficiente (como vamos ver, logaritmo) a um elemento.

\subsubsection{Versão funcional}







\section{Exerc�cios Resolvidos}

\begin{enumerate}

\item .... pesquisa em lista ordenada ....

\item .... pesquisa em arranjo com sentinela ....

\end{enumerate}

\section{Exerc�cios}


