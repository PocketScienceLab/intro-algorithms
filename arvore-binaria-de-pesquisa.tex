\section{Árvore binária de pesquisa}
\label{sec:arvore-binaria-de-pesquisa}

Uma árvore binária de pesquisa é uma árvore binária --- isto é, um
árvore com duas sub-árvores (possivelmente vazias), digamos, à
esquerda e à direita --- com os elementos contidos nos nodos, um
elemento por nodo, e tal que: todo elemento contido em um nodo é maior
que os elementos contidos na sub-árvore à esquerda deste nodo e menor
que os elementos contidos na sub-árvore à direita do nodo.

Dois exemplos de árvores binárias de pesquisa com os elementos de 1 a
7 são mostradas abaixo. 

\begin{verbatim}
      4                    4
     / \                  / \
    1   5                2   6
     \   \              / \  /\
     3    6            1  3 5  7 
    /      \ 
    2       7
\end{verbatim}

A propriedade fundamental de uma árvore binária de pesquisa é o acesso
eficiente (logarítmico no caso médio: veja comentário sobre a
complexidade no caso médio abaixo) a um elemento.

\subsubsection{Versão funcional}

Considere a definição de árvore binária com elementos nos nodos
internos da árvore apresentada na seção \sec{sec:arvores}:

\begin{center}
\begin{tabular}{l}
\begin{hask}{ArvB,FolhaB,NodoB}{White}
 data ArvB a = FolhaB | NodoB a (ArvB a) (ArvB a)
\end{hask}
\end{tabular}
\end{center}

A pesquisa por um elemento em uma árbore de pesquisa binária é um
algoritmo simples, mostrado a seguir:

\begin{center}
\begin{tabular}{l}
\begin{hask}{pesq}{\decremento}
pesq v FolhaB = False
pesq v (NodoB v' t t') 
  | v == v'    = True
  | v <  v'    = pesq v t
  | otherwise  = pesq v t'
\end{hask}
\end{tabular}
\end{center}

O algoritmo para inserção de elementos em uma árvore binária de
pesquisa é semelhante:

\begin{center}
\begin{tabular}{l}
\begin{hask}{ins}{\decremento}
ins v FolhaB = NodoB v FolhaB FolhaB
ins v t@(NodoB v' t1 t2) 
  | v <  v'    = ins v t1
  | v >  v'    = ins v t2
  | otherwise  = t
\end{hask}
\end{tabular}
\end{center}

A complexidade de tempo de \inh{pesq} é determinada a seguir: 

\begin{enumerate}

  \item Vamos considerar que o tamanho da entrada é dado pelo número
    de elementos $n$ na árvore.

  \item As operações relevantes são as comparações de igualdade
    (\inh{==}) e menor-que (\inh{<}). 

  \item \label{complexidade-da-pesquisa-binaria-no-pior-caso} A
    expressão-determinante-da-eficiência é, no pior caso, igual a
    $T(n-1) + k$, onde $k$ é uma constante que expressa o tempo de
    execução das comparações de igualdade e menor-que. No pior caso, o
    elemento está em sub-árvore que contém $n-1$ nodos: todos os nodos
    menos o nodo corrente. Neste caso, a árvore está
    desbalanceada. Isso ocorre, por exemplo, para as árvores mostradas
    a seguir, construídas com inserções sucessivas de elementos em uma
    lista crescente e decrescente, respectivamente.

    \begin{verbatim}

         v0                             v0
          \                             /
          v1                           v1
           \                           /
           ...                        ...
            \                         /
            vn-1                     vn-1
             \                      /
              vn                   vn

    \end{verbatim}

  \end{enumerate}

Assim, a relação de recorrência neste caso é:

   \[ T(n) = T(n-1) + k \]

A solução dessa relação de recorrência foi apresentada na seção
\ref{sec:maior-elemento}:

       \[ T(n) \asymp n \]

Note que o algoritmo de pesquisa em uma árvore binária de pesquisa tem
complexidade $O(p)$ no pior caso, onde $p$ é a profundidade da árvore
(número de arestas do maior caminho existente entre a raiz e uma
folha).

Em uma árvore balanceada, o número de elementos $n$ é igual a $2^p$,
ou seja, em uma árvore balanceada, a pesquisa tem complexidade 

  \[ T(n) \asymp lg n \]
Em uma árvore totalmente desbalanceada, temos no entanto $p = n-1$.

\subsection{Versão imperativa}

Vamos usar a definição de árvore binária apresentada na seção
\ref{sec:arvores}:

\begin{center}
\begin{tabular}{l}
\begin{alg}{ArvoreBinariaDeInteiros}{White}
struct ArvoreBinariaDeInteiros
     int elem
     struct ArvoreBinariaDeInteiros *esq, dir 
\end{alg}
\end{tabular}
\end{center}

A pesquisa iterativa por um valor \ina{k} em uma árvore \in{arvBin} é
apresentada a seguir:

\begin{center}
\begin{tabular}{l}
\begin{alg}{pesq}{\decremento}
pesq(k, arvBin) 
    nodoCorrente §$\from$§ arvBin
    while nocoCorrente != NULL
        v §$\from$§ nocoCorrente->elem
        if v == k 
           return nodoCorrente
        else if k < v
                  nodoCorrente §$\from$§ nodoCorrente->esq
             else nodoCorrente §$\from$§ nodoCorrente->dir
    return NULL
\end{alg}
\end{tabular}
\end{center}

A complexidade é claramente a mesma da versão recursiva. 

Uma árvore binária de pesquisa é uma importante estruturas de dados,
gerando implementações simples e eficientes de pesquisa, inserção e
remoção (em computação, um {\em dicionário\/} é um tipo abstrato que
define tais operações; veja exercício resolvido \ref{ex:dicionario}).

Embora a complexidade da pesquisa a um elemento seja logarítmica no
caso médio, como vimos no item
\ref{complexidade-da-pesquisa-binaria-no-pior-caso} acima, no pior
caso a complexidade da pesquisa é a mesma da complexidade da pesquisa
sequencial, quando a árvore está totalmente desbalanceada, ficando
equivalente a uma lista.

Há muitos trabalhos de pesquisa em computação que procuram preservar o
balanceamento de árvores binárias de pesquisa, para manter a
propriedade de complexidade logarítmica para as operações de inserção,
remoção, pesquisa e ordenação de valores, baseados principalmente em
técnicas de transformação: simplificação ou mudança de representação.

A simplificação transforma árvores não balanceadas em árvores
balanceadas (segundo alguma critério de balanceamento) sem adicionar
nenhuma informação adicional a nodos da árvore. A seção
\ref{sec:arvores-AVL} aborda {\em árvores AVL\/}, nas quais a
diferença entre a altura das sub-árvores de qualquer nodo não deve ser
maior que 1.

Outro estrutura de dados que provê balanceamento de árvores de
pesquisa baseada na técnica de transformação por simplificação é a que
vamos chamar de {\em árvores MovPraRaiz\/} (em inglês, {\em splay
  trees\/}), baseada na ideia de mover para raiz um elemento em
operações de inserção, remoção ou pesquisa.  Árvores MovPraRaiz são
abordadas no exercício resolvido \ref{ex:arvores-MovPraRaiz}.

Exemplos de balanceamento via mudança de representação são {\em
  árvores vermelha-e-preta\/} (também chamadas de rubro-negras, em
inglês {\em red-black trees\/}) e {\em árvores B\/} (em inglês, {\em
  B-trees\/}), abordadas respectivamente nas seções
\ref{sec:arvore-vermelha-e-preta} e \ref{sec:arvores-B}.

\section{Árvores AVL}
\label{sec:arvores-AVL}

....

\section{Árvores Vermelha-e-Preta}
\label{sec:arvore-vermelha-e-preta}

Uma árvorte vermelha-e-preta é uma árvore binária de pesquisa com uma
informação a mais em cada nodo que indica sua cor, que pode ser
vermelha ou preta, e que deve além disso satisfazer às seguintes
condições (chamadas de {\em condições-de-arvore-vermelho-e-preta\/}):

\begin{enumerate}

\item As folhas e a raiz são pretas.

\item Os filhos de todo nodo vermelho são pretos.

\item Todo caminho de um nodo a uma folha contém o mesmo número de
  nodos pretos.

\end{enumerate}

Um exemplo de uma árvore vermelha-e-preta é mostrada abaixo.

.... aqui vem figura de árvore vermelha-e-preta. ....

As duas últimas condições garantem a propriedade {\em
  altura-varia-pouco\/}: o comprimento do caminho mais longo da raiz a
uma folha não é maior que o dobro do comprimento do menor caminho da
raiz a uma folha. Ou seja, a árvore é razoavelmente bem balanceada em
termos de altura dos seus nodos (altura = comprimento do maior caminho
do nodo até uma folha). Como o pior caso da complexidade de tempo de
operações de inserção, remoção e pesquisa são proporcionais à altura
da árvore, isso permite uma eficiência maior, no pior caso, do que
árvores binárias de pesquisa.

Para ver porque as duas últimas condições garantem a propriedade
altura-varia-pouco, considere que $n_p$ seja o número de nodos pretos
de um caminho a uma folha, para uma árvore $a$.  Seja $k$ o
comprimento do menor caminho $c_m$ da raiz a uma folha (esse caminho
contém $n_b$ nodos pretos). O caminho mais longo da raiz a uma folha
$c_M$ não pode conter nodos vermelhos em níveis consecutivos da árvore
(um novo vermelho não pode ser filho de outro nodo vermelho), e o
número de nodos pretos tem que ser igual a $n_p$. Portanto, o número
de nodos de $c_M$ tem que ser no máximo igual a $2*k$ (caso em que
$c_m$ só tem nodos pretos).

Operações que modificam a árvore geralmente causam reorganização dos
nodos e cores dos nodos de modo a que as
condições-de-árvore-vermelho-e-preta continuem a ser satisfeitas, o
que pode ser feito de modo eficiente. 

% The balancing of the tree is not perfect but it is good enough to
% allow it to guarantee searching in O(log n) time, where n is the total
% number of elements in the tree. The insertion and deletion operations,
% along with the tree rearrangement and recoloring, are also performed
% in O(log n) time.

\section{Árvores B}
\label{sec:arvores-B}
