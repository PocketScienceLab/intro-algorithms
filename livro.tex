%!TEX encoding = ISO-8859-1
\documentclass[a4paper,10pt]{book}

\usepackage[T1]{fontenc}             % font encoding
\usepackage[brazil]{babel}           % language
%\usepackage[portuguese]{babel}
\usepackage[latin1]{inputenc}      % acentos

\usepackage{amsmath,amssymb,amsthm}  % mathematics
\usepackage{graphicx}                               % graphics
\usepackage{color}                                     % cores
\usepackage{indentfirst}  % indentation at paragraph beggining
\usepackage{hyperref}     % for links in pdf and webpage
\usepackage{makeidx}
%\usepackage{subfigure}
%\usepackage{comment}

%-- colors --------------------------------------
\definecolor{bluencs}{rgb}{0.0, 0.53, 0.74}
\definecolor{paleblue}{rgb}{0.74, 0.83, 0.9}
%\definecolor{magnolia}{rgb}{0.97, 0.96, 1.0}
\definecolor{darkgreen}{rgb}{0.0,0.5,0.0}
\definecolor{brown}{rgb}{0.65,0.16,0.16}
\definecolor{ngray}{rgb}{0.86, 0.86, 0.86}

%-- hevea and page styling -------------------------
\usepackage{hevea}
\usepackage{hevea-epsfbox-def}  
\usepackage{fancybox}
\usepackage{fancyheadings}
\usepackage{fancysection}            % hevea styling for sections
\colorsections{200}

%--- page style ----------------------------------
\setlength{\headheight}{15.2pt}
\pagestyle{fancyplain}
\addtolength\headwidth{\marginparsep}
\renewcommand{\chaptermark}[1]
             {\markboth{#1}{}}
\renewcommand{\sectionmark}[1]
             {\markright{\thesection\ #1}}
\lhead[\fancyplain{}{\sfseries\thepage}]
     {\fancyplain{}{\sfseries\rightmark}}
\rhead[\fancyplain{}{\sfseries\leftmark}]
      {\fancyplain{}{\sfseries\thepage}}
\cfoot{}

%----- text format parameters -----
\setlength\textwidth{150mm}
\setlength\textheight{241mm}
\setlength\oddsidemargin{5.4mm}
\setlength\evensidemargin{5.4mm}
\setlength\marginparsep{5mm}
\setlength\marginparwidth{15mm}
\addtolength\voffset{-5mm}
\addtolength\topmargin{-3mm}
\addtolength\headsep{2mm}
\setlength\footskip{10mm}


%- typesetting code -------------------------------
\usepackage{listings}                                 

% estilo default de typesetting de c�digo
\newstyle{.myclisting}{font-family:monospace;white-space:pre;margin:lex;padding:2ex;}

% estilo para Haskell
%OBS: a versao original de Haskell considera como 
% keywords todas as defini��es do prelude.base
\lstdefinestyle{mystyle}{basicstyle={\tt},
backgroundcolor={\color{paleblue}},
keywordstyle={\bf \color{black}},
identifierstyle={\color{blue}},
commentstyle={\rm \it \color{red}},
emphstyle={\color{darkgreen}},
xleftmargin=0pt,xrightmargin=0pt,
columns={fixed},
keepspaces={true},
showstringspaces=false,
stepnumber=1,
numbersep=5pt,
numberstyle={\tt \footnotesize \color{white}},
captionpos=b,
escapechar=�
}

% seta estilo global de lstlistings para uso por lstinline
\setenvclass{lstlisting}{myclisting}
\lstset{style=mystyle}

% listings e lstinline para Haskell 
\lst@definelanguage{myHaskell}{language=Haskell,%
deletekeywords={List,Cons,Nil,Pair,Tree,Leaf,Node,%
all,and,any,delete,elem,foldl,foldr,insert,map,max,or,sort}}
{}


\lstnewenvironment{hask}[1]{
\lstset{language=myHaskell,emph={#1}}}
{}

\newcommand{\inh}{\lstinline[language=myHaskell]}

% listings e lstinline para C 
\lst@definelanguage{myAlg}{language={C++},%
morekeywords={with}}

\lstnewenvironment{alg}[1]{
\lstset{language={myAlg},emph={#1}}}
{}

% environment para algoritmo com linhas numeradas
\lstnewenvironment{algno}[1]{
\setenvclass{lstlisting}{myclisting}
\lstset{language={myAlg},numbers=left,emph={#1}}}
{}


\newcommand{\ina}{\lstinline[language=myAlg]}
%-----------------------------------------------

\setcounter{tocdepth}{3}

%-- title------------------------------------------
\title{\textsc{{\color{white} Introdu��o a Algoritmos}}\\
%{\small release ##= version ##}}
\author{Carlos Camar�o\\
{\small Universidade Federal de Minas Gerais}\\
{\small Doutor em Ci\^encia da Computa\c{c}\~ao pela Universidade de Manchester, Inglaterra}\\
{\small $\copyright$ 2015} }
%\date{Febuary 25, 2015}
\pagestyle{headings}

\htmlhead{{\sc\bf \small Introdu��o a Algoritmos}:}

\makeindex


%----- lista de exercicios ---------------
%\newenvironment{exerc}
%   {\renewcommand{\labelenumi}{\textbf{\theenumi.}}
%    \begin{enumerate}
%   }
%   {\end{enumerate}}                    

\begin{document}

\maketitle

� permitida a duplica��o ou reprodu��o, no todo ou em parte, sob
quaisquer formas ou por quaisquer meios (eletr�nico, mec�nico,
grava��o, fotoc�pia, distribui��o na Web ou outros), desde que seja
para fins n�o comerciais.

\tableofcontents

\setcounter{secnumdepth}{-2}
%\chapter{Pref�cio}

Este livro prov� uma introdu��o ao estudo de algoritmos. Ele apresenta
estruturas de dados b�sicas e algoritmos de pesquisa e ordena��o, e
prov� uma introdu��o ao importante ramo da ci�ncia da computa��o que
trata do desenvolvimento e an�lise da efici�ncia de algoritmos. O
assunto da an�lise de efici�ncia � comumente chamado em computa��o de
complexidade.

Essa an�lise aborda em geral quanto tempo � gasto na execu��o de um
algoritmo em fun��o do tamanho da entrada: diz-se complexidade de
tempo do algoritmo. Al�m do tempo, pode ser analisada tamb�m a
complexidade de espa�o (quanto espa�o de mem�ria � gasto na execu��o
em fun��o do tamanho da entrada).

... nota��o ...

... funcional ...

... clareza, concis�o ...

\subsection{Conte�do e Organiza��o do Livro}

\subsection{Recursos Adicionais}

\subsection{Pr�-requisitos}

Os pr�-requsitos s�o:

\begin{enumerate}

\item Experi�ncia inicial com provas por indu��o. (??)

\end{enumerate}


\input{meta.keys}

%\pagestyle{fancy}
\setcounter{secnumdepth}{10}
\pagenumbering{arabic}

%\chapter{Introdu��o}
\label{Introducao}

Este livro prov� uma introdu��o ao estudo de algoritmos. Ele apresenta
uma introdu��o a estruturas de dados b�sicas e algoritmos de pesquisa
e ordena��o.

Vamos identificar neste livro algoritmo com fun��o, no sentido de
prover uma sequ�ncia de passos que associa a cada valor de (um
conjunto de valores de) entrada um �nico valor de (um conjunto de
valores de) sa�da.

A diferen�a que existe entre o conceito usual de fun��o � a nota��o
usualmente empregada para especifica��o da sequ�ncia de passos. Em
computa��o, � usual o emprego de uma nota��o ou linguagem {\em
  imperativa\/}, ao passo que usualmente defini��es de fun��es
empregam uma nota��o mais {\em declarativa\/}, ou {\em funcional\/}.

\section{Ordena��o}

Por exemplo, considere o problema de ordena��o, especificado
formalmente como a seguir (um problema computacional especifica a
rela��o que deve existir entre a entrada e a sa�da):

\Entrada: sequ�ncia de elementos $S_0$.

\Saida: sequ�ncia de elementos ordenada $S$ tal que $S$ � uma
permuta��o de $S_0$.

Uma sequ�ncia $a_1, \ldots, a_n$ � ordenada se $a_i \leq a_{i+1}$ para
$i=1,\ldots, n-1$.

\subsection{Sobre Permuta��o} 

Uma permuta��o (ou arranjo) � uma redisposi��o de (um conjunto ou
sequ�ncia de) elementos em uma certa sequ�ncia (se contrap�e a uma
{\em combina��o}, na qual a ordem dos elementos resultantes n�o �
relevante). Por exemplo, h� 6 permuta��es distintas dos elementos
1,2,3, que s�o, escritas como tuplas: (1,2,3), (1,3,2), (2,1,3),
(2,3,1), (3,1,2), (3,2,1).

Outro nome, usado no contexto de palavras, � {\em anagrama}. 

Um anagrama � o resultado de rearranjar as letras de uma palavra ou
frase para produzir uma nova palavra ou frase, usando todas as letras
originais exatamente uma vez. Por exemplo, "ovo" pode ser rearranjado
para "voo".

Permuta��es ocorrem em diversas �reas da matem�tica e proeminentemente
no estudo de algoritmos, particularmente de ordena��o, em ci�ncia da
computa��o.

O n�mero de permuta��es de $n$ elementos distintos � igual ao fatorial
de $n$ (usualmente escrito em matem�tica como $n!$), que � igual ao
produto de todos os inteiros positivos menores ou iguais a $n$.

\section{Ordena��o por Inser��o}

Um algoritmo ou fun��o que resolve o problema de ordena��o
especificado acima, chamado de {\em ordena��o por inser��o\/}, �
mostrado a seguir. Ele reflete o modo como um jogador de baralho
usualmente ordena uma sequ�ncia de cartas recebidas (por exemplo, em
um jogo de buraco).

Neste livro, usamos sempre pseudo-c�digos com nota��o funcional e
imperativa para representar cada algoritmo. A nota��o funcional � a
linguagem \Haskell\ e a nota��o imperativa � um pseudo-c�digo
semelhante a \C, \Pascal\ ou \Java.

O uso da nota��o funcional pode ser desconsiderado em cursos que
desejam abordar apenas o paradigma de programa��o imperativo.

\subsection{Vers�o funcional}
\label{insertion-sort-func}

A nota��o funcional ser� explicada sempre que necess�rio, isto �,
sempre que houver alguma possibilidade de d�vida. Uma descri��o
sucinta da linguagem Haskell � inclu�da no Ap�ndice \ref{Ap-Haskell}
(o leitor n�o familiarizado com Haskell deve ler o Ap�ndice
\ref{Ap-Haskell}).  Descri��es mais completas de Haskell podem ser
encontradas, por exemplo, em \cite{PeytonJones92,Thompson99}.

\newcommand{\elem}{{\it elem\/}}
\newcommand{\insert}{{\it insert\/}}
\newcommand{\sort}{{\it sort\/}}
\newcommand{\delete}{{\it delete\/}}
\newcommand{\perm}{{\it is\_a\_permutation\_of\/}}
\newcommand{\sorted}{{\it sorted\/}}
\newcommand{\Pair}{{\it Pair\/}}

\progb{
      \sort\ []       \hspace*{1cm} = []\\
      \sort\ ($a$:$x$)\             = \insert\ $a$ (\sort\ $x$)\\
      \hspace*{1cm}\\
      \insert\ $a$ []       \hspace*{1cm} = []\\
      \insert\ $a$ ($b$:$x$) \\ 
          \hspace*{.2cm} | $a$ <= $b$  \hspace*{1.2cm} = $a$: ($b$ : $x$)\\
          \hspace*{.2cm} | \otherwise  \hspace*{.7cm} = $b$: \insert\ $a$ $x$
}

Explica��es sobre a nota��o funcional (usada em Haskell):

\begin{enumerate}

\item $f$ $x$ (aplica��o funcional --- a base da programa��o
  funcional) � o mesmo que {\tt $f$($x$)} (mas melhor porque evita os
  par�nteses).

\item {\tt $b$: \insert\ $a$ $x$} � o mesmo que {\tt $b$: (\insert\ $a$
  $x$}): a aplica��o funcional tem preced�ncia sobre o uso de
  operadores bin�rios.

\item O uso de um operador bin�rio nada mais � do que uma varia��o
  sint�tica de (a��car sint�tico para) uma aplica��o funcional; o uso
  de um operador bin�rio pode ser transformado em uma aplica��o
  funcional, e vice-versa. Para transformar um operador bin�rio em uma
  aplica��o funcional, basta colocar o operador entre par�nteses, e
  para transformar uma aplica��o funcional em um operador, basta
  colocar o nome da fun��o entre crases.
  Exemplos: 

  \begin{tabular}{lll} 
    {\tt 2 + 3} & � equivalente a & {\tt (+) 2 3} \\ 
    {\tt $b$ : $x$} & � equivalente a & {\tt (:) $b$ $x$} \\ 
    $f$ $x$ $y$ & � equivalente a & {\tt $x$ `$f$` $y$} 
  \end{tabular}

\item As fun��es \insert\ and \sort\ usam listas, um tipo recursivo, que
  � um tipo de dado alg�brico (chamado em Haskell de {\tt data})
  parecido com o seguinte:

    \progb{\data\ \List\ $a$ = \Nil\ | \Cons\ $a$ (\List\ $a$)}

  Um tipo de dado alg�brico � a maneira como se definem somas (de
  tipos, sendo que s� podem existir somas disjuntas de tipos), que
  modelam escolha (``ou'') de tipos de dados.

  A declara��o de \List\ acima especifica que um valor de tipo lista �
  polim�rfico (o uso da vari�vel de tipo $a$ indica que \List\ � um
  construtor de tipos que pode ser aplicado a {\em qualquer\/} tipo
  $t$, isto �, podemos ter qualquer inst�ncia \List\ $t$, para {\em
    qualquer\/} tipo $t$), e que uma lista (um valor de tipo
  $\List\ t$, para algum tipo $t$) pode ser \Nil\ (uma lista vazia)
  {\em ou\/} {\tt \Cons\ $v$ $x$}, uma lista (n�o vazia) formada por
  um valor $v$ (cabe�a da lista) e de um restante (ou rabo) da lista,
  $x$ (que deve ser do mesmo tipo da lista da qual � o restante).

  O tipo de listas em Haskell (� parecido mas) difere ligeiramente do
  tipo alg�brico acima porque o construtor \Nil\ � escrito como {\tt
    []} e o construtor \Cons\ � escrito como um operador bin�rio {\tt
    :}. Assim, em vez de escrever, {\tt \Cons\ 1 \Nil}, escreve-se em
  Haskell {\tt 1:[]}. Al�m disso, pode-se escrever tamb�m {\tt
    [1,2,3]} em vez de {\tt 1:2:3:[]} --- i.e.~em vez de {\tt
    1:(2:(3:[]))}.

\item Tipos de dados alg�bricos permitem definir somas (disjuntas) de
  tipos, que modelam escolha (``ou'') de tipos de dados. Para definir
  produtos de tipos, podemos usar tipos alg�bricos, que permitem
  produtos ``linearizados'' (tamb�m chamados de ``currificados'') ou
  produtos cartesianos (generalizados), tamb�m chamados de tuplas. 

  Por exemplo:

  \progb{\data\ \Pair\ $a$ $b$ = \Pair\ $a$ $b$}

  define um construtor de tipos \Pair, que tem dois par�metros que
  podem ser instanciados para quaisquer tipos $t$ e $t'$: por exemplo,
  \Pair\ \Int\ \Bool\ representa pares de valores de inteiros e
  booleanos (o primeiro componente do par � um inteiro e o segundo um
  valor booleano). � semelhante ao produto {\tt (\Int,\Bool)}. A
  diferen�a � que valores do primeiro s�o constru�dos da forma {\tt
    \Pair\ 1 \True} (especificando um valor inteiro e em seguida um
  valor booleano), ao passo que valores do segundo s�o constru�dos da
  forma {\tt (1,\True)} (especificando, entre par�nteses, primeiro um
  valor inteiro, seguido de uma v�rgula, e depois um valor booleano).

\end{enumerate}

Para mostrar a corre��o de \sort, podemos usar predicados (fun��es de
contra-dom�nio \Bool); vamos provar:

  \[ \text{\it \sort\ $x$ = $y$ implica:} \]

  \begin{enumerate}

    \item \sorted\ $y$
    \item $x$ `\perm` $y$

 \end{enumerate}

Temos:

  \progb{
        \sorted\ []            \hspace*{2cm}= \True\\
        \sorted\ [$a$]         \hspace*{1.7cm}= \True\\
        \sorted\ ($a$:$b$:$x$) \hspace*{.1cm}= ($a$ <= $b$) \&\& \sorted\ ($b$:$x$) \\ 
        \hspace*{1cm} \\  
        [] \symbol{96}\perm\symbol{96} []         \hspace*{1.3cm} = \True\\
        ($a$:$x$) \symbol{96}\perm\symbol{96} $y$ \hspace*{.1cm} = \elem\ $a$ $y$ \&\& 
                                ($x$ \symbol{96}\perm\symbol{96} (\delete\ $a$ $y$)) \\
        \hspace*{1cm} \\  
        $a$ \symbol{96}\elem\symbol{96}\ []        \hspace*{1cm} = \False\\
        $a$ \symbol{96}\elem\symbol{96}\ ($b$:$x$) \hspace*{.1cm}= ($a$ == $b$) || ($a$ \symbol{96}\elem\symbol{96} $x$)\\
        \hspace*{1cm} \\  
        \delete\ $a$ []         \hspace*{0.7cm} = []\\
        \delete\ $a$ ($b$:$x$)\\
           \hspace*{.2cm}| $a$==$b$     \hspace*{2cm}  = $x$ \\
           \hspace*{.2cm}| \otherwise\  \hspace*{0.7cm} = $b$: \delete\ $a$ $x$
  }

Explica��es sobre a nota��o funcional (usada em Haskell):

\begin{enumerate}

\item {\tt \&\&} e {\tt ||} s�o operadores bin�rios l�gicos de
  conjun��o e disjun��o, respectivamente.

\item A defini��o de \delete\ usa {\em guardas\/}, que s�o express�es
  booleanas usadas na defini��o de fun��es; para cada chamada de
  fun��o, a primeira (na ordem textual) guarda cuja avalia��o retorna
  \True\ define o resultado da chamada da fun��o, pela avalia��o da
  express�o associada a essa guarda (que segue o s�mbolo {\tt =}). Por
  exemplo, a guarda na defini��o de \delete\ � equivalente a:
  \iif\ $a$==$b$ \tthen\ $x$ \eelse\ $b$: \delete\ $a$ $x$.

\end{enumerate}

Prova: O caso base sai diretamente e o caso indutivo � consequ�ncia
dos seguintes lemas:

Lema 1: Para todo $a,x$, 
        {\tt \sorted ($x$)} implica {\tt \sorted(\insert\ $a$ $x$)}

Lema 2: Para todo $a,x$, 
        {\tt \sort\ $x$ \symbol{96}\perm\symbol{96} $x$} implica 
        {\tt \insert\ $a$ (\sort\ $x$) \symbol{96}\perm\symbol{96} ($a$:$x$)}

% ... prova dos lemas?

\subsection{Vers�o imperativa}
\label{insertion-sort-imperativ}

A vers�o imperativa usa o pr�prio arranjo para a ordena��o (nenhum
outro arranjo ou estrutura de dados auxiliar) e a seguinte ideia:

  \begin{quotation}
     insere {\tt $A$[$j$]} no arranjo ordenado de {\tt $A$[1]} at�
     {\tt $A$[$j$-1]}, de $j=2$ at� o tamanho do arranjo
  \end{quotation}

A ideia d� origem ao seguinte algoritmo, escrito em pseudo-c�digo como
(note que endenta��o no pseudo-c�digo indica aninhamento na estrutura
de blocos):

\newcommand{\key}{{\it key\/}}

\progb{
\sort($A$) \{ \\
\hspace*{.5cm} \for\ j $\leftarrow$ 2 \tto\ \length[$A$] \do\\
\hspace*{1cm}     \key\ $\leftarrow$ $A$[j]\\
\hspace*{1cm}     /* Insere $A$[$j$] no arranjo ordenado $A$[1..$j$-1] */ \\
\hspace*{1cm}     $i$ $\leftarrow$ $j$-1\\
\hspace*{1cm}     \while\ ($i$ > 0 \&\& $A$[$i$]>\key) \do\\
\hspace*{2cm}        $A$[$i$+1] $\leftarrow$ $A$[$i$]\\
\hspace*{2cm}        $i$ $\leftarrow$ $i$ - 1\\
\hspace*{1cm}    $A$[$i$+1] $\leftarrow$ \key\\
\}
}

A corre��o do algoritmo adv�m de que o invariante, especificado
bastante informalmente como:

  \begin{quotation}
    --- no in�cio da execu��o de cada itera��o do comando \for, o
    sub-arranjo {\tt $A$[1..$j$-1]} cont�m os elementos que estavam
    originalmente nesse sub-arranjo, mas de forma ordenada ---
  \end{quotation}

� verdadeiro no in�cio (antes da execu��o da primeira itera��o do
\for), antes e ap�s cada itera��o, e no final, quando ent�o a
termina��o garante a corre��o do algoritmo (ordenamento de todo o
arranjo).

� importante observar que a transforma��o dessa prova informal em uma
prova formal � relativamente muito mais dif�cil do que no caso
funcional.

No pr�ximo cap�tulo vamos introduzir introduzir a nota��o e os
conceitos principais usados para an�lise da complexidade (efici�ncia)
de algoritmos, para que possamos analisar a complexidade de algoritmos
(come�ando pela complexidade dos algoritmos apresentados neste
cap�tulo).



 %!TEX encoding = ISO-8859-1
\chapter{Complexidade}
\label{ch:Complexidade}

Programas de computador est�o sendo usados cada vez mais, nas 
mais diversas �eras. Al�m disso, a ci�ncia da computa��o tem contribu�do e influenciado in�meras �reas cient�ficas, como a l�gica, a matem�tica, a
f�sica etc.

Em computa��o, n�o nos preocupamos apenas com a corre��o de
algoritmos --- o fato de que um algoritmo computa o resultado
especificado, para cada inst�ncia dos dados de entrada ---
mas tamb�m nos precupamos com a sua efici�ncia, ou {\em complexidade}. A A efici�ncia de um algoritmo refere-se, usualmente, ao tempo gasto na execu��o do mesmo, mas pode tamb�m ser referente ao espa�o, ou quantidade de mem�ria, usado durante a sua execu��o.

A efici�ncia, ou complexidade, de um algoritmo � medida como 
uma fun��o de uma medida que especifica o tamanho da entrada. 
Dizemos que o tempo de execu��o de um algoritmo tem ordem de complexidade {\em polinomial}, se a varia��o do seu tempo de execu��o com o tamanho $n$ da entrada � limitado superiormente por uma fun��o polinomial $n^k$, onde $k$ � um valor constante, que independe de $n$. Existem problemas -- considerados {\em dif�ceis} -- para os quais nenhum dos algoritmos conhecidos para sua solu��o tem tempo de execu��o de complexidade polinomial.  Algoritmos conhecidos para esses problemas dif�ceis t�m ordem de complexidade {\em exponencial}, ou seja, tempo de execu��o limitado superiormente por uma fun��o $k^n$, onde $n$
� o tamanho da entrada e $k$ � um valor constante, que independe de
$n$. 


\section{Complexidade de Fun\c{c}\~oes}
\label{sec:complexidade-de-funcoes}

O tempo de execu��o de um algoritmo � medido em termo do n�mero de passos de execu��o do algoritmo. Cada passo de execu��o corresponde a uma instru��o ou opera��o at�mica, relevante no problema em quest�o, e que � executada em um intervalo de tempo constante. Por exemplo, o tempo de execu��o de um algoritmo de compara��o, tal como o apresentado na Se��o \ref{sec:ordenacao-insercao}, pode ser medido como o n�mero de compara��es entre elementos da sequ�ncia, at� que a sequ�ncia seja ordenada. Essa maneira de medir o tempo de execu��o abstrai do tempo espec�fico gasto para execu��o de um passo, permitindo a compara��o de efici�ncia de algoritmos independentemente da m�quina em que s�o executados.

O tempo de execuc�o de uma sequ�ncia de $k$ passos � a soma dos tempos de execu��o de cada passos, ou, mais simplesmente, o n�mero tde passos $k$. A execu��o de determinadas constru��es usadas na defini��o de algoritmos, como comandos de repeti��o ou chamadas de fun��es recursivas, resultam na execu��o de uma determinada sequ�ncia de passos repetidas vezes. Se uma sequ�ncia de $k$ passos � executada $n$ vezes, o tempo total de execu��o �, naturalmente, $n\times k$.

Na maioria das vezes, estamos interessados em determinar o tempo de
execu��o do algoritmo para o {\em pior caso\/} de uma entrada de certo
tamanho. Isso � devido ao fato de que i) o pior caso � um limite
superior (que nunca poder� ser ultrapassado), ii) para muitos
algoritmos o pior caso ocorre bastante frequentemente e iii)
frequentemente o pior caso representa valor pr�ximo do caso m�dio, e
iv) em geral, o {\em caso m�dio\/} � mais dif�cil de ser analisado,
envolvendo t�cnicas de an�lise de probabilidades.

Al�m disso, em geral estamos interessados em determinar como se comporta a fun��o de complexidade de um algoritmo � medida que a entrada tende para tamanhos muiro grandes. Por exemplo, se a fun��o que expressa a varia��o do
tempo de execu��o com o tamanho da entrada � $f(n) =
an^2 + bn + c$, onde $a,b,c$ s�o constantes, dizemos que essa fun��o tem
ordem de complexidade $n^2$, que � o termo mais significativo do
polin�mio $f$. Isso porque os termos de menor ordem no polin�mio s�o relativamente pouco significantes para valores de $n$ grandes. Al�m disso, ignora-se tamb�m a constante $a$ do termo de maior ordem, pelo mesmo motivo de ser relativamente pouco significante para valores grandes. Dizemos
que essa fun��o $f$ � de complexidade quadr�tica (o termo de
maior ordem no polin�mio que define a complexidade � $n^2$). Em outras palavras, estamos interessados no comportamento {\em assint�tico\/} da fun��o de complexidade do algooritmo. A nota��o geralmente usada paradescrever a {\em complexidade assint�tica\/} de algoritmos � apresentada a seguir.


%\section{Efici�ncia assint�tica}
%\label{sec:eficiencia-assintotica}

%Ao considerar a varia��o da efici�ncia de acordo com tamanhos de
%entrada grandes, fazemos simplifica��es para estudar a efici�ncia ou
%complexidade de algoritmos, e chamamos a complexidade de {\em
%  assint�tica\/}. Usualmente, considera-se algoritmos mais eficientes
%que outros considerando a complexidade assint�tica.


\section{Nota��o $\Theta$}
\label{sec:Notacao-Theta}

O tamanho da entrada de um algoritmo � medido como um n�mero natural natural -- i.e.~o dom�nio de fun��es de complexidade � igual ao conjunto dos naturais, $\mathbb{N}$) e o contra-dom�nio � o conjunto dos n�meros reais
positivos ($\mathbb{R}^+$. 

Seguimos a terminologia usual, que usa $f(n)$ (digamos, por exemplo,
$lg n$) para se referir na verdade � fun��o $f$ (em nota��o de
$\lambda$-calculus, � fun��o $\lambda n \rightarrow lg n$). Temos:

\[ \begin{array}{ll}
     \Theta(g(n)) = \{ f(n) : & \text{ existem } c_1, c_2, n_0 \text{ positivos tais que } \\
                              & c_1g(n) \leq f(n) \leq c_2g(n) \text{ para todo } n\geq n_0 
   \end{array}
\]

Em palavras, $f(n)$ � membro de $\Theta(g(n))$ se $f(n)$ sempre est�
entre $c_1g(n)$ e $c_2g(n)$, para determinados $c_1,c_2$, e para $n$
suficientemente grande (i.e.~a partir de algum $n_0$). Isso � ilustrado na Figura \ref{fig:big-O}(a). 


\begin{figure}
\label{fig:big-O}
\begin{center}
\imgsrc[width=700]{lvimages/fig-bigO.jpg}
\end{center}
\caption{Big-$\Theta$, Big-O, Big-$\Omega$ notations}
\end{figure}


Como $\Theta(g(n))$ � um conjunto, o correto seria escrever $f(n) \in
\Theta(g(n))$, mas � usual escrever $f(n) = \Theta(g(n))$, um abuso de
nota��o, que simplifica a nota��o. 
Escrevemos tamb�m $f(n) \asymp g(n)$ (seguindo Minko Markov
\cite{Minko-Markov-2013}) como sin�nimo de $f(n) = \Theta(g(n))$.

%ssume-se tamb�m, na defini��o de $\Theta(g(n))$ que $g(n)$ e os
%embros $f(n)$ s�o assintoticamente n�o-negativos para $n$
%uficientemente grande.

Por exemplo, temos:

\begin{enumerate}

\item $an^2 - bn \asymp n^2$, para quaisquer constantes
      $a,b$.

      Para mostrar isso, devemos determinar $c_1,c_2,n_0$ tais que $0
      \leq c_1(n^2) \leq an^2 - bn \leq c_2(n^2)$, para $n\leq n_0$.
      Ou seja (dividindo por $n^2$):

        \[ c_1 \leq a -\frac{b}{n} \leq c_2 \text{, para } n\leq n_0 \]
      Essa desigualdade pode ser satisfeita, para todo $n\geq 1$,
      tomando $c_1 \leq a - b$ e $c_2 \geq a$. 

\item $an \not\asymp n^2$, para qualquer constante positiva $a$.

      Nesse caso, dever�amos ter $c_1n^2 \leq an$, para $n$
      suficientemente grande, ou seja, $c_1n \leq a$, o que n�o
      acontece.

\item $an^3 \not\asymp n^2$, para qualquer constante positiva $a$.

      Nesse caso, dever�amos ter $c_2n^2 \geq an^3$, para $n$
      suficientemente grande, ou seja, $c_2 \leq an$, o que n�o
      acontece.

\end{enumerate}

$\Theta(1)$ � usualmente usado em vez de $\Theta(n^0)$,
considerando-se claro, pelo contexto a vari�vel usada como medida do tamanho da entrada.

\section{Nota��o $O$}
\label{sec:Notacao-O}

A nota��o $\Theta$ limita assintoticamente uma fun��o por limites
superior e inferior. A nota��o $O$ estabelece apenas um limite
superior:

\[ \begin{array}{ll}
     O(g(n)) = \{ f(n) : & \text{ existem } c, n_0 \text{ positivos tais que } \\
                              & f(n) \leq cg(n) \text{ para } n\geq n_0 
   \end{array}
\]

Em palavras, $f(n)$ � membro de $O(g(n))$ se $f(n)$ sempre � menor ou
igual a $cg(n)$, para algum $c$, e para $n$ suficientemente grande
(i.e.~a partir de algum $n_0$). A Figura \ref{fig:big-O}(b) ilustra essa defini��o. 
 
Assim como na nota��o $\Theta$, usa-se $f(n) = O(g(n))$ em vez de
$f(n) \in O(g(n))$. 

Escrevemos tamb�m $f(n) \preceq g(n)$ como sin�nimo de $f(n) =
O(g(n))$ (seguindo Minko Markov \cite{Minko-Markov-2013}).

Note que $f(n) \asymp g(n)$ implica $f(n) \preceq g(n)$, mas a
implica��o inversa n�o � verdadeira.

Por exemplo, $an + b \preceq n^2$ mas $an + b \npreceq \Theta(n^2)$.

Usando a nota��o $O$ podemos frequentemente ter uma boa ideia do
limite superior para o tempo de execu��o de um programa, pela inspec��o
de sua estrutura de repeti��o.  Por exemplo, o aninhamento duplo do
programa imperativo de ordena��o por inser��o, apresentado na Se��o \ref{sec:ordenacao-insercao}, indica um limite
superior $O(n^2)$ para o pior caso do tempo de execu��o. O
custo de cada comando interno ao comando \while\ � $O(1)$, as
repeti��es \for\ e \while\ s�o controladas pelos �ndices $i$ e $j$, que
variam no m�ximo at� $n$, e a repeti��o mais interna � executada no
m�ximo uma vez para cada par de $n^2$ valores $(i,j)$.

Note que a nota��o $O$ fornece um limite superior e, portanto, � v�lida
para qualquer entrada. Isso n�o ocorre com a nota��o $\Theta$, uma vez
que podem existir entradas para as quais o algoritmo se comporta mais
eficientemente. Por exemplo, $O(n^2)$ � v�lido para qualquer entrada do algoritmo de ordena��o por inser��o (� um limite superior), mas existem entradas para as quais o tempo de execu��o � linear, dado por $\Theta(n)$, especificamente, se a entrada j� est� ordenada. 

Dizer que o tempo de execu��o do algoritmo de ordena��o por inser��o �
$O(n^2)$ significa, portanto, que existe uma fun��o $f$, em $O(n^2)$,
tal que, para qualquer entrada e qualquer $n$, o tempo de execu��o do
programa para essa entrada � limitado a $f(n)$ -- n�o significa que a
varia��o do tempo de execu��o do algoritmo para uma entrada particular
varia quadraticamente com o tamanho da entrada, mas que pode variar no
m�ximo quadraticamente com o tamanho da entrada.

\section{Nota��o $\Omega$}
\label{sec:Notacao-Omega}

A nota��o $\Omega$ limita assintoticamente uma fun��o por um limite
inferior:

\[ \begin{array}{ll}
     \Omega(g(n)) = \{ f(n) : & \text{ existem } c, n_0 \text{ positivos tais que } \\
                              & cg(n) \leq f(n) \text{ para } n\geq n_0 
   \end{array}
\]

Em palavras, $f(n)$ � membro de $\Omega(g(n))$ se $f(n)$ sempre �
maior ou igual a $cg(n)$, para algum $c$, e para todo $n$
suficientemente grande. Isso � ilustrado na Figura \ref{fig:big-O}(c)

Assim como na nota��o $\Theta$, usa-se $f(n) = \Omega(g(n))$ em vez de
$f(n) \in \Omega(g(n))$.

Escrevemos tamb�m $f(n) \succeq g(n)$ como sin�nimo de $f(n) =
\Omega(g(n))$ (seguindo Minko Markov \cite{Minko-Markov-2013}).

Das defini��es, � f�cil ver que, para quaisquer fun��es $f,g$, temos
$f(n) \asymp g(n)$ se e somente se $f(n) \preceq g(n)$ e $f(n) \succeq
g(n)$.

A nota��o $\Omega$ estabelece um limite inferior, e envolve, assim,
an�lise do comportamento do algoritmo para o melhor caso dos dados de
entrada. N�o consideraremos an�lises de comportamento de algoritmos no
melhor caso neste livro, e a nota��o $\Omega$ ter� assim aplica��o
limitada, sendo usada mais como complementa��o � nota��o $O$.

\section{Uso de nota��o assint�tica em f�rmulas}
\label{sec:uso-notacao-assintotica}

Em equa��es do tipo $n = O(n^2)$, a nota��o $O$ significa $n \in
O(n^2)$. 

Em geral, no entanto, a ocorr�ncia da nota��o assint�tica expressa uma
fun��o qualquer, an�nima, para a qual n�o h� interesse em especificar
um nome. Por exemplo, $an^2 + bn + c$ pode ser expressa como $an^2 +
\Theta(n)$, siginificando $an^2 + f(n)$, onde $f(n)$ � um membro de
$\Theta(n)$, no caso $bn + c$. Continuando nesse sentido, $an^2 +
\Theta(n)$ pode ser expressa como $\Theta(n^2)$. 

Essas abrevia��es s�o usadas para evitar ter que escrever em f�rmulas fun��es que correspondem a termos de menor grau.

Como exemplo de uso de nota��o assint�tica, temos: $O(2^{O(lg
  n)})=n^{O(1)}$.  O uso da nota��o $O$ nessa f�rmula expressa
supress�o de constantes. Note que $n \preceq 2^{lg n}$, e portanto,
para qualquer constante $c$, temos: $n^c \asymp 2^{c lg n}$, e $O(2^{O(lg
  n)})=n^{O(1)}$ � outra forma de expressar essa igualdade.

\section{Nota��es $o$, $\omega$}
\label{sec:Notacao-o}

Define-se $o(g(n))$ para indicar que trata-se de uma aproxima��o
assint�tica estrita:

\[ \begin{array}{ll}
     o(g(n)) = \{ f(n) : & \text{ para todo } c \text { positivo existe } n_0 \text{ positivo tal que } \\
                              & f(n) < cg(n) \text{ para } n\geq n_0 
   \end{array}
\]

Em palavras, $f(n)$ � membro de $O(g(n))$ se $f(n)$ sempre �
estritamente menor que $cg(n)$, para todo $c$, e para $n$
suficientemente grande (i.e.~a partir de algum $n_0$).
Isso � equivalente a:

  \[ \llim_{n\rightarrow\!\infty} \frac{f(n)}{g(n)} = 0 \]

Analogamente, a nota��o $\omega$ indica um limite inferior que �
assintoticamente estrito ($\omega$ est� para $\Omega$ assim como $o$
est� para $O$):

\[ \begin{array}{ll}
     \omega(g(n)) = \{ f(n) : & \text{ para todo } c \text { positivo existe } n_0 \text{ positivo tal que } \\
                              & cg(n) < f(n) \text{ para } n\geq n_0 
   \end{array}
 \]

Isso � equivalente a:

  \[ \llim_{n\rightarrow\!\infty} \frac{g(n)}{f(n)} = 0 \]

Escrevemos tamb�m $f(n) \succ g(n)$ como sin�nimo de $f(n) = o(g(n))$,
e $f(n) \prec g(n)$ como sin�nimo de $f(n) = \omega(g(n))$ (seguindo
Minko Markov \cite{Minko-Markov-2013}).

Por exemplo, para qualquer $a>0$, $an^2 \succ n$ mas $an^2 \nsucc
n^2$.

\section{Propriedades de Rela��es assint�ticas}
\label{sec:Propriedades-de-relacoes-assintoticas}

Para todas as rela��es $R$ iguais a $\asymp$, $\preceq$, $\succeq$,
$\prec$, $\succ$, a seguinte transitividade ocorre:

  \[ f(n) R g(n), g(n) R h(n) \text { implicam } f(n) R h(n) \]

Ocorre tamb�m reflexividade para $R = \asymp$, $\preceq$, $\succeq$:

  \[ f(n) R f(n) \]

Para $\asymp$ ocorre simetria (mas n�o para as demais rela��es):

  \[ f(n) \asymp g(n) \text{ se e somente se } g(n) \asymp f(n) \]

Para $\succeq$ e $\preceq$, e $\succ$ e $\prec$, temos:

  \[ \begin{array}{l}
       f(n) \succeq g(n) \text{ se e somente se } g(n) \preceq f(n) \\
       f(n) \succ g(n) \text{ se e somente se } g(n) \prec f(n) 
     \end{array}
  \]

Essas propriedades s�o similares �s verificadas para as rela��es de
igualdade e desigualdade entre n�meros reais, motivam o uso das
nota��es semelhantes para fun��es e motivam chamar $f$ de
assintoticamente menor que $g$ se $f(n) = o(g(n))$, $f$ de
assintoticamente maior que $g$ se $g(n) = \omega(f(n))$), e
analogamente para ($\asymp$ e $\Theta$, assintoticamente igual),
($\preceq$ e $O$, assintoticamente menor ou igual a) e ($\succeq$ e
$\Omega$, assintoticamente maior ou igual a).

\section{Complexidade polinomial, exponencial e logar�tmica}
\label{sec:Complexidade-polinomial-exponencial-logaritmica}

Um polin�mio de grau $k$, sendo $k$ inteiro n�o negativo, � uma fun��o
da forma:

  \[ \sum_{i=0}^{k} a_in^i \]
onde $a_i$ s�o constantes inteiras, chamadas de {\em coeficientes\/}
do polin�mio, e $a_k > 0$. 

Seja $k$ uma constante inteira positiva. Dizemos, de uma fun��o $f$
sobre os naturais, que:

  \begin{itemize}
    \item $f$ tem limite (superior) de complexidade polinomial se
      $f(n) = O(n^k)$,
    \item tem limite (inferior) de complexidade exponencial se $f(n) =
      \Omega(k^n)$,
    \item tem limite (superior) de complexidade logar�tmica se $f(n) =
      O((lg n)^k)$.
  \end{itemize}

% $lg^k n$ � abrevia��o de $(lg n)^k$.

$lg$ indica logaritmo na base 2; usamos $log$ para especificar a base,
  de forma que $lg n = log_2 n$. A base de logaritmos em complexidade
  algor�tmica n�o � muito relevante, pois, para quaisquer constantes
  inteiras $a,b,c>0$, temos: $log_b a = \frac{log_c a}{log_c b}$.

$lg lg k$ � abrevia��o de $lg (lg k)$, e $lg$ tem pouca preced�ncia
  (s� se aplica ao pr�ximo termo em uma f�rmula): $lg n + k$ significa
  $(lg n) + k$.

Para quaisquer constantes inteiras $a,b$, se $a>1$ temos:

  \begin{align} 
       \llim_{n\rightarrow\!\infty} \frac{n^b}{a^n} &= 0 \label{expxpoli}  \\
            \llim_{n\rightarrow\!\infty} \frac{lg^b n}{n^a} &= 0 \label{exlxpoli}
     \end{align}

A equa��o \ref{expxpoli} indica que toda fun��o exponencial 
com uma base maior que 1 cresce mais rapidamente que qualquer fun��o polinomial.

A segunda equa��o \ref{exlxpoli} (que vale tamb�m se $a=1$) indica 
que toda fun��o logar�tmica cresce mais lentamente que qualquer fun��o polinomial.

%Lembre-se:

%  \[ e^x \begin{array}[t]{l}
%         = \llim_{n \rightarrow \infty} (1 + \frac{x}{n})^n \\
%         = \sum_{i=0}^\infty \frac{x^i}{i!}
%         \end{array}
%  \]


\section{Exerc�cios Resolvidos}

\begin{enumerate}

\item $2^{n+1} \preceq 2^n$ ? 

Resposta: Sim. Pois $2^{n+1} = 2*2^n$. Constantes n�o interferem na
ordem de complexidade (na defini��o de $O$, basta escolher a constante
$c$ adequadamente, neste caso podemos escolher qualquer $c\geq 2$).

\item $2^{2n} \preceq 2^n$ ? 

Resposta: N�o.

Suponha que sim. Dever�amos ter ent�o $c,n_0$ tais que $0\leq 2^{(2n)}
\leq c2^n$, para $n\geq n_0$. Dividindo por $2^n$ --- note:
$2^{2n}=(2^n)^2$ --- obtemos: $2n \leq c$, para todo $n\geq n_0$, o
que � falso.

A fun��o que recebe $n$ e retorna $2^{2n}$ � chamada de duplamente
exponencial.

\item Usa-se $\lfloor x \rfloor$ (o ch�o de $x$) para denotar o menor
  inteiro maior ou igual a $x$, e $\lceil x \rceil$ (o teto de $x$)
  para denotar o maior inteiro menor ou igual a $x$.

  A fun��o $\lfloor lg n \rfloor$ tem limite de complexidade
  polinomial?

Resposta: Sim. 

$\lfloor x \rfloor < x + 1$, e $lg n \preceq n$, e portanto $\lfloor
lg n \rfloor \preceq n$.

\item Cada entrada da tabela abaixo indica, para o par formado por $v
  = f(n)$ e $w = g(n)$ das duas primeiras colunas da tabela, se $f(n)
  = X(g(n))$, para $X$ variando de acordo com o indicado nas colunas
  seguintes (i.e.~para $X = \Theta,O,\Omega,o,\omega$).  Insira {\tt
    '*'} quando a entrada da tabela abaixo for "Sim", e deixe em
  branco caso contr�rio.
% Sempre que inserir {\tt '*'}, justifique (mostre porque sim).

Suponha $k\geq 1$, $a > 0$ e $b > 1$ constantes.


\newcommand{\hsC}{\hspace*{1.8cm}}

\[ \begin{array}{|c|c|c|c|c|c|c|}
v           & w            & \Theta & O      & \Omega & o &   \omega \\ \hline\hline
(lg n)^k & n^a        &  \hsC    & \hsC & \hsC      & \hsC  &  \hsC \\ \hline
n^k       & b^n        &         &  &        &   &        \\ \hline
\sqrt n  & n^{sin n} &         &  &        &   &        \\ \hline
2^n       & 2^{n/2}  &         &  &        &   &        \\ \hline
n^{lg b} & b^{lg n}  &         &  &        &   &        \\ \hline
lg (n!)    & lg (n^n)  &         &  &        &   &        
    \end{array}
\]

Resposta (baseada em \cite{Minko-Markov-2013}):

\[ \begin{array}{|c|c|c|c|c|c|c|}
    v        & w      & \Theta & O      & \Omega  & o       & \omega\\ \hline\hline
(lg n)^k & n^a        &       & {\tt *} &         & {\tt *} &     \\ \hline
n^k      & b^n        &       & {\tt *} &         & {\tt *} &     \\ \hline
\sqrt n  & n^{\sin\ n} & \hsC    & \hsC & \hsC      & \hsC  &  \hsC \\ \hline
2^n      & 2^{n/2}     &       &         & {\tt *} &         & {\tt *} \\ \hline
n^{lg b}  & b^{lg n}    & {\tt *} & {\tt *} & {\tt *} &        &        \\ \hline
lg (n!)  & lg (n^n) &          & {\tt *} &        & {\tt *} &        
    \end{array}
\]

A primeira linha � consequ�ncia de $\llim_{n\rightarrow\!\infty}
\frac{(lg n)^k}{n^a} = 0$, para todo $k$ e todo $a>0$.

A segunda linha � consequ�ncia de $\llim_{n\rightarrow\!\infty}
\frac{n^k}{b^n} = 0$, para todo $a$ e todo $b>1$.

$\sin\ n$ oscila (continuamente) entre 0 e 1, quando $n$ cresce de 0 a
$\infty$. Portanto, $2^{1/2}$ e $2^{\sin\ n}$ n�o se relacionam (com as
rela��es assint�ticas acima).

$\frac{2^{n/2}}{2^n} = \frac{1}{2^{n/2}}$, e portanto $2^n \preceq
2^{n/2}$.

Temos: $\begin{array}[t]{lll}
          lg (n^{lg b}) & = lg b (lg n) & \asymp (lg n) \\
          lg (b^{lg n}) & = lg n (lg b) & = \asymp (lg n) 
        \end{array}$ e, portanto, $n^{lg b} \asymp b^{lg n}$. 

Temos: 

\[ \begin{array}[t]{llllllll}
n!  = n & \times (n-1) & \times (n-2) & \times (n-3) & \times \ldots 
                                      & \times 3 & \times 2 & \times 1\\
n^n = n & \times  n    & \times  n    & \times n     & \times \ldots
                                      & \times n & \times n & \times n
\end{array}
\]

As duas linhas t�m exatamente $n$ termos, e cada termo do lado direito
da primeira � menor que o termo correspondente da segunda linha, para
$n$ suficientemente grande.  Logo, $n! \leq n^n$ para $n$
suficientemente grande.

\item Ordene as seguintes fun��es sobre os naturais por ordem de
  complexidade. Ou seja, ordene as fun��es de $f_1$ a $f_{30}$ de modo
  que $f_i \prec f_{i+1}$ ou $f_i \asymp f_{i+1}$, para $i=1,\ldots,29$.

  As fun��es, aplicadas a $n$, s�o: 

   \[ \begin{array}{|c|c|c|c|c|c|}
 lg (lg n) & (\sqrt{2})^{lg n}  & n^2          & (3/2)^n       & n^3         & lg^* n        \\
 2^{2^n}    & n^{(\frac{1}{lg n})}  & n^n          & lg n          & 2^{lg n}     & (lg n)^{lg n}  \\
 2^n       & 4^{lg n}           & n lg n       & 2^{2^{n+1}}     & n!          & (lg n)!       \\
 (n+1)!    & lg (n!)           & lg (lg^* n)  & 2^{lg * n}     & lg^* (lg n) & 2^{\sqrt{2 lg n}} \\
 n^{lg lg n} & 1                 & \sqrt{lg n}  & n            & n 2^n       & (lg n)^2
      \end{array}
   \]

A nota��o $lg^*$ � definida a seguir.  Considere primeiro que
$f^{(i)}$ denota a fun��o ``f aplicada $i$ vezes'', para $i\geq 0$:

\[ f^{(i)} x = 
    \left\{ \begin{array}{ll}
      x            & se $i=0$ \\
      f(f^{(i-1)} x) & caso contr�rio
    \end{array}\right. 
\]

$lg^*$ � uma fun��o que recebe um argumento $n$ e retorna o menor $i$
tal que $lg^{(i)} n \leq 1$, ou, em outras palavras, retorna quantas
vezes se precisa aplicar $lg$ para obter-se 1 ou menos:

  \[ lg^* n = min \{ i \mid i\geq 0, lg^{(i)} n \leq 1 \} \]

Por exemplo, $\begin{array}[t]{llll} 
lg^* 2          &             & = 1 & (lg^{(0)} 2 = 2, lg^{(1)} 2 = lg (lg^{(0)} 2) = lg 2 = 1) \\
lg^* 3          &             & = 2 & (lg^{(0)} 3 = 3, lg^{(1)} 3 = lg 3, lg^{(2)} 3 = lg (lg 3) = 0.6644\ldots) \\
lg^* 2^2        & = lg^* 4     & = 2 & (lg^{(0)} 4 = 4, lg^{(1)} 4 = lg 4 = 2, lg^{(2)} 4 = lg 2 = 1) \\
lg^* 5          &             & = 3 & (lg^{(0)} 5 = 5, lg^{(1)} 5 = lg 5 = 2.3219\ldots, lg^{(2)} 5 = lg 2.3219\ldots = 1.2153\ldots, \ldots)\\
\ldots          &             &     &       \\
lg^* 2^{2^2}     & = lg^* 16    & = 3 & \\ 
               & = lg^* 17    & = 4 & \\
\ldots         &              &     & \\ 
lg^* 2^{2^{2^2}}  & = lg^* 65536 & = 4 & \\
\ldots         & = lg^* 65537 & = 5 & \\
\ldots         &              &     & \\
lg^* 2^{2^{2^{2^2}}} & = \ldots     & = 5 & \\
\ldots         &              &       
              \end{array}$

Ou seja, $lg^*$ cresce {\em muito\/} lentamente. S� poder�amos
escrever o pr�ximo valor ($ n_6 = 2^{2^{2^{2^{2^2}}}} $) usando
exponencia��o: $ n_5 = 2^{2^{2^{2^2}}} $ tem 19729 d�gitos, mas $n_6$
tem um n�mero de d�gitos extraordin�rio, ``maior do que o n�mero de
�tomos ($\approx 10^{82}$) que se estima existir no universo que
podemos observar'' (i.e.~o universo que se expande at� 90 e poucos
bilh�es de anos-luz; $10^{82}$ � da ordem de (menor que)
$2^{(10/3)\times 82}$, que � muito menor que $2^{65536}$). Note que
$10^3$ � aproximadamente igual (um pouco menor que) $2^{10}$, e
portanto $10^k = 10^{3 \times (k/3)} = (10^3)^{k/3}$, que � portanto
da ordem de $(2^{10})^{k/3} = 2^{10 \times (k/3)} = 2^{(10/3)\times
  k}$.

Solu��o:

\begin{enumerate}

\item $1 \asymp n^{\frac{1}{lg n}}$

Isso pode ser mostrado tomando $lg$ de $n^{\frac{1}{lg n}}$. Temos que
$lg (n^{\frac{1}{lg n}}) = (\frac{1}{lg n}) \times lg n = 1$ (e
portanto $n^{\frac{1}{lg n}} = 2$).  
Logo, $1 \asymp n^{\frac{1}{lg n}}$.

\item $1 \prec lg (lg^* n)$

Direto. Pois $1 \prec lg (lg^* n)$ decorre de $1 < 2 * lg (lg^* 4)$
--- pela defini��o das rela��es $(\prec)$ e $O$, tomando $n_0 = 4, c = 2$ e usando o
fato de que $lg (lg^* n)$ � monot�nica, i.e.~cresce ou continua igual
quando $n$ cresce. O que � v�lido pois: 
 $1 < 2 * lg (lg^* 4)$ � o mesmo que
 $1 < 2 * lg 2$, ou seja, $1 < 2$.

\item $lg (lg^* n) \prec lg^* n$

Seja $m = lg^* n$. Temos ent�o que provar: $lg m \prec m$.

$lg m \prec m$ � consequ�ncia de $\llim_{n\rightarrow\!\infty} \frac{(lg n)}{n} = 0$.

\HRule
{\em Nota\/}: 
O fato de que esse limite � zero � conhecido, mas pode ser obtido
usando a regra de l'H�pital, como a seguir.

Usando $'$ (diz-se: 'linha') para denotar a derivada de uma fun��o, a
regra de l'H�pital especifica que, para todas as fun��es $f$, $g$
diferenci�veis em um intervalo aberto $I$ exceto possivelmente em um
ponto $k \in I$, se i) $\llim_{x \to k}f(x)=\lim_{x \to k}g(x)=v$,
onde $v = 0$ ou $v = \pm\infty$, ii) $\lim_{x\to
  k}\frac{f '(x)}{g'(x)}$ existe, e iii) $g'(x)\neq 0$ para todo $x\in
I - \{ k\}$, ent�o $\lim_{x\to k}\frac{f(x)}{g(x)} = \lim_{x\to
  c}\frac{f '(x)}{g'(x)}$.

Assim, para todo $k>0$, temos: 
  $\lim_{n\rightarrow\!\infty} \frac{(lg n)}{n^k}$ � igual a 
  $\lim_{n\rightarrow\!\infty} \frac{(\frac{ln n}{ln 2})}{n^k}$ 
que, pela regra de l'H�pital, usando o fato de que 
  $(ln n)' = \frac{1}{n}$ 
e $(n^k)' = k \times n^{k-1}$, 
obtemos 
  $\lim_{n\rightarrow\!\infty} \frac{(\frac{1}{n \times (ln 2) \times k})}{1}$, 
que � igual a 0. Ou seja, para todo $k>0$ temos:
  \begin{equation}  
    \lim_{n\rightarrow\!\infty} \frac{(lg n)}{n^k} = 0 
               \label{eq:lim-poli-sobre-exp}
  \end{equation} 

Para mostrar que a derivada de $ln$ � a fun��o inversa (i.e.~$(ln n)'= \frac{1}{n}$),
seja: $y = ln x$, ou seja, $e^y = x$; derivando ambos os lados em
rela��o a $x$, temos: $e^y (\frac{dy}{dx}) = 1$, ou seja, 
 $x (\frac{dy}{dx}) = 1$, isto �: $\frac{dy}{dx} = \frac{1}{x}$.

{\em Fim de Nota\/} 
\HRule

\item $lg^* n \asymp lg^* (lg n)$

Veja a varia��o de valores da fun��o $lg^*$ (veja explica��o acima). A
diferen�a entre $lg^* n$ e $lg^* (lg n)$ � $1$, ou seja, $lg^* (lg n)
= (lg^* n) - 1$.

\item $lg^* n \prec 2^{lg^* n}$

Consequ�ncia de $m \prec 2^m$ (fazendo $m = lg^* n$). 

\item $2^{lg^* n} \prec lg (lg n)$

Para poder comparar mais facilmente, eliminamos a exponencia��o
tomando o logaritmo (aplicando $lg$) aos dois lados. Obtemos: $lg^* n
\prec lg (lg (lg n))$. Como $lg^* n$ s� cresce com o n�mero de
pot�ncias de 2 de uma torre de pot�ncias de 2 que expressa o valor de
$n$, e como $lg 2^i = i$, com $n$ a partir de $n_5 = 2^{2^{2^{2^2}}}$,
para o qual $lg^* n_5 = 5$ e $lg (lg (lg n_5)) = 2^{2^2} = 16$,
teremos sempre $lg^* n$ menor que $lg(lg(lg n))$.

\item $lg (lg n) \prec \sqrt{lg n}$

Com $lg n = m$ obtemos $lg m \prec m^{\frac{1}{2}}$, que �
consequ�ncia de (\ref{eq:lim-poli-sobre-exp}).

\item $\sqrt{lg n} \prec lg n$

Com $lg n = m$ obtemos $\sqrt m \prec m$, o que � verdadeiro (na
defini��o de O, basta escolher $n_0 = 1, c = 1$).

\item $lg n \prec (lg n)^2$

Com $lg n = m$ obtemos $m \prec m^2$, o que � verdadeiro (na defini��o
de O, basta escolher $n_0 = 1, c = 1$).

\item $(lg n)^2 \prec 2^{\sqrt{2 lg n}}$

Com $lg n = m$ obtemos $m^2 \prec 2^{\frac{m}{2}}$, i.e.~$4m^2 \prec 2^m$, 
o que � consequ�ncia de $\lim_{n\rightarrow\!\infty} \frac{n^b}{a^n} = 0$, 
para todas as constantes $a$ e $b$ tais que $a>1$. 

\item $2^{\sqrt{2 lg n}} \prec \sqrt{2}^{lg n}$

Temos: $\sqrt{2}^{lg n} = 2^{\frac{lg n}{2}}$. Assim, usando o fato de
que, para todo $k,f,g$, $k^{f(n)} \prec k^{g(n)}$ se e somente se
$f(n) \prec g(n)$, obtemos o resultado desejado se e somente se
$\sqrt{2 lg n} \prec \frac{lg n}{2}$, o que � verdadeiro pois
$\sqrt{lg n} \prec lg n$.

\item $\sqrt{2}^{lg n} \prec n$

Temos: $\sqrt{2}^{lg n}  \prec n$ se e somente se 
       $2^{\frac{lg n}{2}} \prec n$ se e somente se 
       $2^{lg \sqrt{n}}   \prec n$ se e somente se 
       $\sqrt{n}       \prec n$,
o que � verdadeiro. % (na defini��o de $O$, basta escolher $n_0 = 4$, $c = 1$).

\item $2^{lg n} \asymp n$

Aplicando $lg$, obtemos: $lg n \asymp lg n$.

\item $n \prec n lg n$.

Podemos escolher por exemplo $n_0 = 1$, $c=2$ na defini��o de $O$.

\item $n lg n \asymp lg (n!)$

Usando a {\em aproxima��o de Stirling\/}:

  \[ n! = \sqrt{2\pi n} (\frac{n}{e})^n (1 + \Theta(\frac{1}{n})) \]
e aplicando $lg$ a ambos os lados, obtemos: $lg (n!) \asymp lg (\sqrt{2\pi n}) + n lg n - n lg e$, 
que � assintoticamente igual a $n lg n$:

  \begin{equation}
    lg (n!) \asymp n lg n 
    \label{lgn-eq-nlgn}
  \end{equation}

\item $n lg n \prec n^2$

Podemos escolher por exemplo $n_0 = 2$, $c=1$ na defini��o de $O$.

\item $n^2 \asymp 4^{lg n}$

Pois $4^{lg n} = (2^2)^{lg n} = 2^{2^{lg n}} = 2^{2 lg n} = 2^{lg (n^2)} = n^2$. 

\item $n^2 \prec n^3$

Podemos escolher por exemplo $n_0 = 1$, $c=2$ na defini��o de $O$.

\item $n^3 \prec (lg n)!$

Aplicando $lg$ a ambos os lados, obtemos $lg (n^3) \prec lg ((lg n)!)$. 
Com $m = lg n$, obtemos: $3 m \prec lg (m!)$ (pois $lg (n^3) = 3 lg n$). 

Usando (\ref{lgn-eq-nlgn}) obtemos: $3 m \prec m lg m$, que significa
$3 \prec lg m$, que � verdadeiro.

\item $(lg n)! \prec (lg n)^{lg n}$. 

� equivalente a $m!\prec m^m$, com $m = lg n$. O que � verdadeiro, pois:

  \[ \lim \frac{m \times (m-1) \times \ldots 2 \times 1}
               {m \times m \times \ldots m \times m} = 0 
  \]
  
\item $(lg n)^{lg n} \asymp n^{lg (lg n)}$

Vamos usar o fato de que $log_b a^n = n log_b a$, para todo $a,b,n$.

Aplicando $lg$ ao lado esquerdo, obtemos: 

 \[ lg ((lg n)^{lg n}) = lg n \times lg (lg n) \]

Aplicando $lg$ ao lado direito, obtemos:

  \[ lg (n^{lg (lg n)}) = lg (lg n) \times lg n \]

\item $n^{lg (lg n)} \prec (\frac{3}{2})^n$

Aplicando $lg$ ao lado esquerdo, obtemos: $lg n \times lg (lg n)$. 

Aplicando $lg$ ao lado direito, obtemos: $n lg \frac{3}{2}$. 

Temos $n \prec lg (lg n)$ e $lg (lg n) \prec lg n \times lg (lg n)$.

Por transitividade da rela��o $(\prec)$, obtemos: $n \prec lg n \times
lg (lg n)$, e portanto $n lg \frac{3}{2} \prec lg n \times lg (lg n)$.

O resultado � ent�o obtido pelo fato de que:

  \begin{equation}
    \text{Para toda fun��o} f,g, \text{ temos: }
        lg f(n) \prec lg g(n) \text{ se e somente se } f(n) \prec g(n) 
    \label{lgprec}
  \end{equation}

\item $(\frac{3}{2})^n \prec 2^n$

Consequ�ncia de: $\lim_{n\rightarrow\!\infty} \frac{(\frac{3}{2})^n}{2^n} = \lim_{n\rightarrow\!\infty} (\frac{3}{4})^n = 0$.

\item $2^n \prec n 2^n$

Com $m = 2^n$ ($n = lg m$), obtemos: $m \prec m lg m$.

\item $n 2^n \prec n!$

Temos: $2^n \prec n!$ e, por transitividade, uma vez que $n 2^n \prec
2^n$, obtemos $n 2^n \prec n!$.

\item $n! \prec (n+1)!$

Consequ�ncia de: $(n+1)! = (n+1) \times (n!)$.

\item $(n+1)! \asymp n^n$

Aplicando $lg$ a ambos os lados, obtemos, usando (\ref{lgn-eq-nlgn}):
  $(n+1) lg (n+1) \asymp n lg n$, o que � verdadeiro.
Obtemos o resultado usando (\ref{lgprec}).

\item $n^n \prec 2^{2^n}$

Aplicando $lg$ a ambos os lados, obtemos, usando (\ref{lgn-eq-nlgn}):
  $n lg n \prec lg (2^n)$, o que � verdadeiro.
Obtemos o resultado usando (\ref{lgprec}).

\item $2^{2^n} \prec 2^{2^{n+1}}$

Temos: $2^{n+1} = 2 \times 2^n$ e portanto $2^{2^{n+1}} = 2^{2\times
  {2^n}} = 2^{2^n} \times 2^{2^n}$.

\end{enumerate}

\end{enumerate}

\section{Exerc�cios}

\begin{enumerate}

\item \ldots

\end{enumerate}



%\chapter{Estruturas de Dados B�sicas}
\label{estruturas-de-dados-basicas}

Este cap�tulo aborda listas e �rvores e suas representa��es e
opera��es b�sicas.

\section{Listas}
\label{listas}

Uma estrutura de dados b�sica em computa��o, chamada de lista, �
definida recursivamente como a seguir.  Uma lista de elementos de
determinado tipo $t$ � ou i) vazia ou ii) um elemento de tipo $t$ e um
resto, sendo o resto tamb�m uma lista de tipo $t$.

Em uma linguagem como Haskell, que prov� suporte � defini��o de tipos
de dados recursivos, o tipo lista pode ser definido como:

  \[ \text{{\tt data \List\ $a$ = \Nil\ | \Cons\ $a$ (\List\ $a$)}} \]

Al�m de recursivo, \List\ $a$ � um tipo polim�rfico: a vari�vel de
tipo $a$ pode ser instanciada para um tipo qualquer, permitindo assim
a defini��o de listas com elementos de um tipo qualquer. 

Al�m disso, a linguagem Haskell prov� suporte � cria��o de valores de
tipo lista por meio do uso de uma nota��o especial para cria��o de
listas, simplesmente colocando os elementos entre colchetes, separados
por v�rgulas --- por exemplo {\tt [1,2,3]} representa a lista {\tt
  \Cons\ 1 (\Cons\ 2 (\Cons\ 3 \Nil))}.

De fato, {\tt []} � usado em vez de \Nil, e o construtor infixado {\tt
  (:)}, associativo � direita, em vez de \Cons; por exemplo, {\tt 1 :
  2 : 3 : []} denota o mesmo valor representado, pela defini��o acima,
com os construtores \Nil\ e \Cons, por {\tt \Cons\ 1 (\Cons\ 2
  (\Cons\ 3 \Nil))}.

Em linguagens como \C, um tipo lista distinto tem que ser definido
para cada tipo distinto para os elementos, usando registros (chamados
em \C\ de "estruturas", definidos com a palavra-chave \struct) e
ponteiros, formando uma estrutura de dados comumente chamada de {\em
  lista encadeada}. Por exemplo:

\newcommand{\eelem}{{\it elem\/}}
\newcommand{\ListaDeInteiros}{{\it ListaDeInteiros\/}}

  \progb{
        \struct\ \ListaDeInteiros\ \{ \\
        \hspace*{.2cm} int \eelem; \\
        \hspace*{.2cm} \struct\ \ListaDeInteiros\ * $r$; \\
        \}
  }
define um registro com dois campos, um campo de tipo {\tt int} e nome
\eelem, e um campo de nome $r$ que � um ponteiro para valores do
pr�prio tipo \ListaDeInteiros.

A manipula��o de valores de tipo lista em \C\ � mais trabalhosa, pouco
elegante e leg�vel e sujeita a erros. Por exemplo, simplesmente para
criar um valor de tipo lista igual a {\tt [1,2,3]} em Haskell, �
necess�rio c�digo como o seguinte:

\progb{\struct\ \ListaDeInteiros\ *$p$ = \malloc(\sizeof(\struct\ \ListaDeInteiros));\\
      $p$->\eelem\ = 1;
      $p$->$r$ = \malloc(\sizeof(\struct\ \ListaDeInteiros));\\
      $p$->$r$->\eelem\ = 2;
      $p$->$r$->$r$ = \malloc(\sizeof(\struct\ \ListaDeInteiros));\\
      $p$->$r$->$r$->\eelem\ = 3;
      $p$->$r$->$r$->$r$ = \NULL;
     }

A falta de suporte a manipula��o de valores de tipos recursivos e
polim�rficos torna a programa��o mais dif�cil e demorada e o c�digo
menos leg�vel e mais sujeito a repeti��es e a ocorr�ncias de erros.

Em Haskell, o acesso a um valor $v$ de uma lista $x$ requer acesso a
todos os elementos anteriores a $v$ em $x$. De fato, a representa��o
de listas � feita em Haskell por meio de ponteiros, mas a manipula��o
de ponteiros � gerada automaticamente, de acordo com o c�digo Haskell
usado, em vez de ser feita diretamente pelo programador.

O uso de arranjos � uma maneira alternativa de representar listas,
especialmente em uma linguagem (como \C) que n�o prov� suporte a
manipula��o de valores de estruturas de dados recursivas como
listas. Nesse caso, o uso de arranjos requer que se especifique a
priori um n�mero m�ximo de elementos.

\subsection{Pesquisa}
\label{pesquisa-em-lista}

Em computa��o, {\em pesquisar\/} em geral significa determinar se um
dado elemento est� presente ou n�o em uma estrutura de dados. As
subse��es seguintes tratam de opera��es de pesquisa, inser��o e
remo��o de elementos de listas, de acordo com a forma com que listas
s�o representadas.

Em ambos os casos apresentados abaixo, a pesquisa em uma lista de $n$
elementos tem complexidade $O(n)$ no pior caso, pois envolve
possivelmente compara��o com cada elemento da lista.

\subsubsection{Vers�o funcional}

\newcommand{\map}{{\it map\/}}

A fun��o \elem\ pode ser definida como a seguir:

\progb{
\elem\ \_ [] = \False\\
\elem\ $a$ ($b$:$x$) \\
  \hspace*{.2cm} | $a$ == $b$  \hspace*{.2cm} = \True\\
  \hspace*{.2cm} | \otherwise\                = \elem\ $a$ $x$
}

O tipo de \elem\ �: {\tt \Eq\ $a$ => $a$ -> $a$ -> \Bool}.  

Esse � um tipo {\em polim�rfico restrito\/}, no qual a restri��o (em
ingl�s, {\em constraint\/}) \Eq\ $a$ indica que a vari�vel de tipo $a$
n�o pode ser instanciada para qualquer tipo, mas apenas para um tipo
que � membro da classe de tipos \Eq, ou seja, no caso apenas para um
tipo para o qual exista definida uma opera��o de igualdade para
valores do tipo. � um erro de tipo chamar \elem\ com um valor para o
qual n�o existe uma defini��o de igualdade definida.

\HRule
{\em Nota\/}: 

A fun��o \elem, de fato, faz parte do m�dulo \Prelude, importado
automaticamente por todos os m�dulos (sem necessidade de comando ou
cl�usula expl�cita de importa��o). A defini��o de \elem\ no
\Prelude\ � feita usando \any, que por sua vez usa \mmap\ e
\oor\ (\oor\ usa \foldr), como a seguir (\mmap\ e \foldr\ s�o ferramentas
importantes para defini��o de fun��es em Haskell):

\progb{
\foldr\ \hspace*{2cm}  :: ($a$ -> $b$ -> $b$) -> $b$ -> [$a$] -> $b$\\
\foldr\ $f$ $z$ []        \hspace*{.8cm} =  $z$\\
\foldr\ $f$ $z$ ($a$:$x$) \hspace*{.1cm} =  $f$ $a$ (\foldr\ $f$ $z$ $x$)\\ 
  \hspace*{.5cm} \\

\mmap\               \hspace*{2cm} :: ($a$ -> $b$) -> [$a$] -> [$b$]\\
\mmap\ \_ []         \hspace*{.8cm} = []\\
\mmap\ $f$ ($a$:$x$) \hspace*{.1cm} = $f$ $a$ : \mmap\ $f$ $x$ \\ 
\hspace*{.5cm}\ \\

\aand, \oor\   :: [\Bool] -> \Bool\\
\aand\        =  \foldr\ (\&\&) \True\\
\oor\          =  \foldr\ (||) \False\\ 
\hspace*{.5cm} \\

\any, \all\ :: ($a$ -> \Bool) -> [$a$] -> \Bool\\
\any\ $p$   =  \oor\ . \map\ $p$\\
\all\ $p$   =  \and\ . \map\ $p$\\ 
\hspace*{.5cm}\ \\

\elem\  \hspace*{.2cm} :: (\Eq\ $a$) => $a$ -> [$a$] -> \Bool\\
\elem\ $a$ = \any\ (== $a$)

}
\HRule

\subsubsection{Vers�o imperativa}

A vers�o imperativa de \elem\ definida a seguir recebe um argumento em
$a$ e um apontador para uma lista (de tipo \ListaDeInteiros), denotado
pelo par�metro $l$, e retorna um apontador para o elemento da lista
igual a $a$, se o argumento estiver presente na lista, e \NULL\ em
caso contr�rio.

  \progb{\elem ($a$, $l$) \{ \\
   \hspace*{.2cm} \while\ ($l$ != \NULL\ \&\& $l$->\elem\ != $a$) $l$ = $l$->$r$;\\
   \hspace*{.2cm} return $l$; \\
   \}
  }

\subsection{Inser��o}
\label{insercao-em-lista}

Inserir um elemento no in�cio de uma lista � uma opera��o de
complexidade $O(1)$.

\subsubsection{Vers�o funcional}

A vers�o funcional cria uma nova lista, que tem como resto a lista
fornecida como argumento, ou seja, � simplesmente igual a {\tt (:)}:

\progb{\insert\ = (:)}

\subsubsection{Vers�o imperativa}

A vers�o imperativa aloca um novo registro, de tipo \ListaDeInteiros,
para inserir um novo valor de tipo inteiro na lista que � fornecida
como argumento. A lista modificada, com a inser��o do novo elemento no
in�cio, � retornada como resultado da opera��o.

\progb{\insert\ ($a$, $l$) \{ \\
         \hspace*{.2cm} $p$ = \malloc(\sizeof(\struct\ \ListaDeInteiros));\\
         \hspace*{.2cm} $p$->\elem\ = $a$;\\
         \hspace*{.2cm} $p$->$r$ = $l$;\\
         \hspace*{.2cm} return $p$;\\
       \}
      }

\subsection{Remo��o}
\label{remocao-de-lista}

Remover um elemento de uma lista � uma opera��o de complexidade $O(n)$
no pior caso, pois � necess�rio procurar o elemento a ser removido.

A Se��o \ref{lista-duplamente-encadeada} redefine o tipo de lista
encadeada para uma vers�o de listas duplamente encadeadas, e reescreve
as fun��es \elem\ e \insert, para definir \delete\ por meio de uma
chamada � fun��o \elem\ (seguida de chamada a \insert).

\subsubsection{Vers�o funcional}

A vers�o funcional cria uma nova lista, que n�o tem o elemento
passado como par�metro:

\progb{
\delete\  \hspace*{1.7cm} :: ($a$ -> $a$ -> \Bool) -> $a$ -> [$a$] -> [$a$]\\
\delete\ \_ [] \hspace*{.2cm} = []\\
\delete\ $a$ ($b$:$x$) \hspace*{.5cm} \\
  \hspace*{.2cm} | $a$ == $b$ \hspace*{.2cm} = $x$\\
  \hspace*{.2cm} | \otherwise\ = $b$ : \delete\ $a$ $x$
}

\subsubsection{Vers�o imperativa}

\newcommand{\prev}{{\it prev\/}}

A vers�o imperativa n�o cria nova lista, usa ponteiro \prev\ para
guardar ponteiro para o elemento a ser removido, e usa ponteiro para o
in�cio da lista tanto como par�metro quanto resultado da opera��o.

\progb{\delete\ ($a$, $l$) \{ \\
    \hspace*{.2cm} \struct\ \ListaDeInteiros\ *\prev\ = \NULL,  *$p$ = $l$;\\
    \hspace*{.2cm} \while\ ($p$ != \NULL \&\& $p$->\eelem\ != $a$)  \\
       \hspace*{1cm} \prev\ = $p$; \\
       \hspace*{1cm} $p$ = $p$->$r$;\\
    \hspace*{.2cm} \iif\ ($p$ != \NULL)\\
       \hspace*{1cm} \iif\ (\prev\ == \NULL)\\ 
	  \hspace*{2cm} \return\ $l$->$r$;\\
       \hspace*{1cm} \eelse\ \{ \prev->$r$ = $p$->$r$; \return\ $l$; \}
  }

\subsection{Lista duplamente encadeada}
\label{lista-duplamente-encadeada}

\ldots \ldots .....

\subsection{Pilha}
\label{pilha}

Uma estrutura de dados {\em Pilha\/} se caracteriza pelo fato de
inser��o, acesso e remo��o de elementos serem feitos apenas em apenas
um de seus lados (ou extremidades). Isso implica em uma pol�tica
algumas vezes chamada de LIFO ({\em last-in first-out\/}: o �ltimo a
ser inserido � o primeiro a ser removido da pilha.

Uma pilha, com opera��es de i) criar pilha vazia, ii) empilhar
elemento, iii) desempilhar elemento, iv) obter elemento do topo da
pilha, e v) testar se pilha est� vazia, pode ser implementada como a seguir.

\newcommand{\vazia}{{\it vazia\/}}
\newcommand{\empilhar}{{\it empilhar\/}}
\newcommand{\desempilhar}{{\it desempilhar\/}}
\newcommand{\topo}{{\it topo\/}}
\newcommand{\estaVazia}{{\it estaVazia\/}}
\newcommand{\elems}{{\it elems\/}}
\newcommand{\pilha}{{\it pilha\/}}
\newcommand{\Pilha}{{\it Pilha\/}}

\subsubsection{Vers�o funcional}

Em Haskell uma lista prov� diretamente as opera��es de uma pilha:

  \progb{\vazia\ = []\\
        \empilhar\ = (:)\\
        \desempilhar\ (\_:$x$) = $x$\\
        \topo\ ($a$:\_) = $a$\\
        \estaVazia\ = \null
       }
                  
Em geral, devemos procurar simplificar o c�digo de nossos programas, e
podemos n�o tratar casos de erro por motivos did�ticos, mas em
programas completos n�o podemos esquecer de tratar todos os casos
poss�veis para os dados de entrada. Um m�dulo em Haskell que trata
todos esses casos poss�veis para os dados de entrada das opera��es
acima envolvendo uma pilha, � mostrado na Figura \ref{fig-Pilha}.

O m�dulo \Pilha\ implementa o que � chamado em computa��o de um {\em
  tipo abstrato de dados\/}, que � um tipo com opera��es de cria��o,
modifica��o e consulta sobre valores desse tipo. Por exemplo,
\vazia\ � uma (de fato, nesse caso a �nica) opera��o de cria��o,
\empilhar\ e \desempilhar\ s�o opera��es de modifica��o, e \topo\ e
\estaVazia\ s�o opera��es de consulta. 

O que caracteriza uma defini��o de um tipo abstrato de dados � o fato
de as opera��es (para cria��o, modifica��o e consulta) sobre valores
do tipo serem definidas e estarem em um trecho de programa (em geral,
no m�dulo) onde o tipo foi definido, mas a representa��o n�o �
conhecida para quem usa valores do tipo. Ou seja, um tipo abstrato
pode ser considerado como um tipo em conjunto com todas as opera��es
poss�veis sobre valores desse tipo. Qualquer opera��o sobre um valor
do tipo s� pode ser realizada por meio dessas opera��es.

Em Haskell, para definir um tipo abstrato pilha usamos um novo tipo,
com um �nico construtor, que n�o � exportado pelo m�dulo em que o tipo
� definido (o construtor de valores \Pilha\ n�o � exportado, apenas o
construtor de tipos \Pilha). Tamb�m definimos, por quest�o de
legibilidade, os tipos de todas as fun��es definidas. Veja Figura
\ref{fig-Pilha}.

\begin{figure}

\progb{
\module\ \Pilha\ (\Pilha, \vazia, \empilhar, \desempilhar, \topo, \estaVazia) \where\\ \ \hspace*{.2cm} \\

\newtype\ \Pilha\ $a$ = \Pilha\ [$a$]\\ \ \hspace*{.2cm} \\ 

\vazia:: \Pilha\ $a$\\
\vazia\ = \Pilha []\\ \ \hspace*{.1cm} \\ \ \hspace*{.2cm} \\

\empilhar:: $a$ -> \Pilha\ $a$ -> \Pilha\ $a$\\
\empilhar\ \eelem\ (\Pilha\ $p$) = \Pilha\ ($e$:$p$)\\ \ \hspace*{.2cm} \\

\desempilhar:: \Pilha\ $a$ -> \Pilha\ $a$\\
\desempilhar\ (\Pilha\ [])    = error "Fun��o desempilhar chamada com pilha vazia"\\
\desempilhar\ (\Pilha\ (\_:$p$)) = \Pilha\ $p$\\ \ \hspace*{.2cm} \\

\topo:: \Pilha\ $a$ -> $a$\\
\topo\ (\Pilha\ [])   = error "Fun��o topo chamada com pilha vazia"\\
\topo\ (\Pilha\ (\eelem:\_)) = \eelem\\ \ \hspace*{.2cm} \\ 

\estaVazia:: \Pilha\ $a$ -> \Bool\\
\estaVazia\ (\Pilha\ $p$) = \null\ $p$
}

\label{fig-Pilha}
\caption{Tipo abstrato \Pilha\ em Haskell}
\end{figure}

\subsubsection{Vers�o imperativa}

Considerando \pilha\ como um registro com campos \topo\ e \elems,
sendo \elems\ um arranjo de $n$ elementos --- indexado de $0$ a {\tt
  $n$-1} --- e \topo\ uma vari�vel inteira que indica o �ndice do
�ltimo elemento inserido, as opera��es em uma pilha podem ser
implementadas como a seguir (desconsiderando casos de erro:
desempilhar de uma pilha vazia e empilhar em uma pilha cheia).

\newcommand{\nome}{{\it nome\/}}

O comando \with\ serve para tornar os nomes de campos de um registro
vis�veis: \with\ $r$ evita que se tenha que prefixar os nomes dos
campos do registro $r$ com o registro (\nome\ pode ser usado em vez de
{\tt $r$.\nome}).

\progb{
  \vazia\ (\pilha) = \{ \pilha.\topo\ = -1; \}\\ \ \hspace*{.2cm} \\
  \empilhar\ (\eelem, \pilha) \{ \\
      \hspace*{.2cm} with \pilha\  \\
        \hspace*{1cm} \topo\ $\leftarrow$ \topo\ + 1;\\
        \hspace*{1cm} \elems[\topo] $\leftarrow$ \eelem;\\
  \} \\ 
  \desempilhar\ (\pilha) \{ \\
      \hspace*{.2cm} with \pilha\ \{ \topo\ $\leftarrow$ \topo\ - 1; \} \\
  \}\\
  \topo\ (\pilha) \{ with \pilha\ \{ \return\ \elems[\topo]; \} \}\\ \ \hspace*{.2cm} \\
  \estaVazia\ (\pilha) \{ \return\ \pilha.\topo\ == -1; \}
 }

\subsection{Fila}
\label{fila}

Em uma {\em fila} a inser��o � feita de um lado e a remo��o � feita do
outro lado da estrutura de dados. Isso implica em uma pol�tica algumas
vezes chamada de FIFO ({\em first-in first-out\/}: o primeiro a ser
inserido � o primeiro a ser removido da fila.

Uma fila, com opera��es de i) criar fila vazia, ii) entrar (inserir
elemento) na fila, iii) sair (tirar elemento) da fila, iv) obter
elemento do in�cio da fila, e v) testar se fila est� vazia, pode ser
implementada como a seguir.

\subsubsection{Vers�o funcional}

\newcommand{\frente}{{\it frente\/}}
\newcommand{\tras}{{\it tr�s\/}}
\newcommand{\fila}{{\it fila\/}}

N�o � eficiente fazer acesso ao �ltimo elemento de uma lista em
Haskell. A implementa��o padr�o de filas por meio de listas usa assim
duas listas, \frente\ e \tras. Elementos entram na lista \tras\ e saem
na lista \frente.

A fun��o \fila\ � usada para garantir o invariante de que se
\frente\ est� vazia, ent�o \tras\ est� vazia (e portanto a fila est�
vazia).

\newcommand{\reverse}{{\it reverse\/}}
\newcommand{\sair}{{\it sair\/}}
\newcommand{\entrar}{{\it entrar\/}}
\newcommand{\inicio}{{\it in�cio\/}}

\progb{\vazia\ = ([],[])\\ \ \hspace*{.2cm} \\
      \entrar\ $e$ (\frente,\tras) = \fila\ (\frente, $e$ : \tras)\\ \ \hspace*{.2cm} \\
      \fila\ ([], \tras) = (\reverse\ \tras, [])\\
      \fila\ $f$        = $f$\\ \ \hspace*{.2cm} \\
      \sair\ ($e$:\frente, \tras) = (\frente,\tras)\\ \ \hspace*{.2cm} \\
      \estaVazia\ (\frente,\_) = \null\ \frente\\ \ \hspace*{.2cm} \\
      \inicio\ ($e$:\frente,\_) = $e$
     }

\subsubsection{Vers�o imperativa}
\label{fila-imperativa}

\newcommand{\fim}{{\it fim\/}}

Considerando \fila\ como um registro com campos \inicio, \fim\ e
\elems, sendo \elems\ um arranjo de $n$ elementos --- indexado de $0$
a {\tt $n$-1}.  Os �ndices do primeiro e do �ltimo elementos inseridos
s�o armazenados respectivamente nas vari�veis \inicio\ e \fim. 

A fila est� vazia quando {\tt \inicio\ == \fim}. A fila est� cheia
quando {\tt \inicio\ == \fim\ + 1}, isto �, a fila � circular: o
�ndice {\tt 0} segue o �ndice {\tt $n$-1}. 

As opera��es em uma fila podem ser implementadas como a seguir,
desconsiderando casos de erro: sair de uma fila vazia e entrar em uma
fila cheia. O operador {\tt \%} � usado para retornar o resto da
divis�o do primeiro operando pelo segundo.

\progb{
  \vazia\ (\fila) \{ with \fila\ \{ \inicio\ = \fim\ = 0 \} \} \\ \ \hspace*{.2cm} \\
  \entrar\ (\eelem, \fila) \{ \\
      \hspace*{.2cm} with \fila\  \\
          \hspace*{1cm} \elems[\fim] $\leftarrow$ \eelem\\
          \hspace*{1cm} \fim\ $\leftarrow$ (\fim\ + 1) \% $n$\\
  \} \\ 
  \sair\ (\fila) \{ \\
      \hspace*{.2cm} with \fila\ \{ \inicio\ $\leftarrow$ (\inicio\ + 1) \% $n$ \}\\
  \}\\
  \inicio\ (\fila) \{ with \fila\ \{ \return\ \elems[\inicio] \} \}\\ \ \hspace*{.2cm} \\
  \estaVazia\ (\pilha) \{ with \fila\ \{ \return\ \inicio\ == \fim \} \} 
 }

%!TEX encoding = ISO-8859-1
\section{�rvores}
\label{sec:arvores}
\index{�rvores}

Uma �rvore � um {\em grafo} conexo e ac�clico. 

\HRule
{\em Nota\/}: 

Um {\em grafo\/} � um conjunto de v�rtices e de arestas, cada aresta
conectando dois v�rtices. Dois v�rtices $a$ e $b$ de um grafo s�o
chamados {\em adjacentes\/} se est�o conectados por uma aresta --
usualmente representada como um par $(a,b)$. Um grafo � {\em conexo\/}
se todo v�rice � adjacente a algum outro. Um {\em caminho\/} de um
v�rtice $a$ a um v�rtice $b$ � uma sequ�ncia de v�rtices adjacentes,
tendo $a$ como primeiro e $b$ como �ltimo v�rtice. Um {\em ciclo\/} �
um caminho que inicia e termina no mesmo v�rtice, n�o repetindo
nenhuma aresta. Um grafo � {\em ac�clico\/} se n�o cont�m nenhum
ciclo. Um grafo ac�clico, mas n�o for conexo, � uma floresta, isto �,
um conjunto de �rvores.

\HRule

No entanto, essa defini��o � baseada no conceito de grafo, que n�o �
abordado neste livro. 

Al�m disso, e mais importante: vamos considerar apenas o que podemos
chamar de {\em �rvore com raiz identificada\/} (em ingl�s, {\em rooted
  tree\/}. O adjetivo {\em identificada\/} procura evitar estranheza,
ao pensar em o que seria uma �rvore sem raiz. Uma �rvore com raiz
identificada pode ser definida recursivamente: � ou i) vazia (uma
folha) ou ii) um nodo contendo um elemento e um certo n�mero de ramos
(ou nodos), que cont�m sub-�rvores. Em Haskell, podemos definir:

\begin{center}
\begin{tabular}{l}
\begin{hask}{Arvore,Folha,Nodo}{White}
data Arvore a = Folha | Nodo a [Arvore a]
\end{hask}
\end{tabular}
\end{center}

Na defini��o acima, o construtor \inh{Folha} constr�i uma �rvore
vazia, ou folha, e o construtor \inh{Nodo} constr�i uma �rvore
contendo um valor de tipo \inh{a} e um lista de sub�rvores.

Outra poss�vel defini��o considera que a informa��o � armazenada nas
folhas, em vez de nos nodos internos:

\begin{center}
\begin{tabular}{l}
\begin{hask}{Avore,Folha,Nodo}{White}
data Arvore' a = Folha' a | Nodo' [Arvore' a]
\end{hask}
\end{tabular}
\end{center}

Uma �rvore com exatamente duas sub-�rvores (possivelmente vazias) �
chamada de �rvore bin�ria e pode ser definida como:

\begin{center}
\begin{tabular}{l}
\begin{hask}{ArvB,FolhaB,NodoB}{White}
 data ArvB a = FolhaB | NodoB a (ArvB a) (ArvB a)
\end{hask}
\end{tabular}
\end{center}

Em linguagens que prov�em suporte ao uso de ponteiros, mas n�o �
defini��o e manipula��o direta de tipos recursivos, a representa��o de
�rvores bin�rias pode ser feita com o uso de ponteiro como mostra o
exemplo a seguir:

\begin{center}
\begin{tabular}{l}
\begin{alg}{ArvoreBinariaDeInteiros}{White}
struct ArvoreBinariaDeInteiros
     int elem
     struct ArvoreBinariaDeInteiros *esq, dir 
\end{alg}
\end{tabular}
\end{center}

Os campos \ina{esq} e \ina{dir} de um nodo s�o ponteiros, possivelmente nulos,
para sub-�rvores.

Para �rvores n�o bin�rias, pode ser usada a seguinte representa��o, que
podemos chamar de {\em representa��o com primog�nito-irm�o-e-pai}:

\begin{center}
\begin{tabular}{l}
\begin{alg}{ArvoreDeInteiros}{White}
struct ArvoreDeInteiros
   int elem;
   struct ArvoreDeInteiros *primogenito, irmao, pai 
\end{alg}
\end{tabular}
\end{center}

%A representa��o da �rvore:

%xxxx

%� mostrada abaixo:

%yyyyy




%%!TEX encoding = ISO-8859-1
\section{�rvores}
\label{sec:arvores}
\index{�rvores}

Uma �rvore � um {\em grafo} conexo e ac�clico. 

\HRule
{\em Nota\/}: 

Um {\em grafo\/} � um conjunto de v�rtices e de arestas, cada aresta
conectando dois v�rtices. Dois v�rtices $a$ e $b$ de um grafo s�o
chamados {\em adjacentes\/} se est�o conectados por uma aresta --
usualmente representada como um par $(a,b)$. Um grafo � {\em conexo\/}
se todo v�rice � adjacente a algum outro. Um {\em caminho\/} de um
v�rtice $a$ a um v�rtice $b$ � uma sequ�ncia de v�rtices adjacentes,
tendo $a$ como primeiro e $b$ como �ltimo v�rtice. Um {\em ciclo\/} �
um caminho que inicia e termina no mesmo v�rtice, n�o repetindo
nenhuma aresta. Um grafo � {\em ac�clico\/} se n�o cont�m nenhum
ciclo. Um grafo ac�clico, mas n�o for conexo, � uma floresta, isto �,
um conjunto de �rvores.

\HRule

No entanto, essa defini��o � baseada no conceito de grafo, que n�o �
abordado neste livro. 

Al�m disso, e mais importante: vamos considerar apenas o que podemos
chamar de {\em �rvore com raiz identificada\/} (em ingl�s, {\em rooted
  tree\/}. O adjetivo {\em identificada\/} procura evitar estranheza,
ao pensar em o que seria uma �rvore sem raiz. Uma �rvore com raiz
identificada pode ser definida recursivamente: � ou i) vazia (uma
folha) ou ii) um nodo contendo um elemento e um certo n�mero de ramos
(ou nodos), que cont�m sub-�rvores. Em Haskell, podemos definir:

\begin{center}
\begin{tabular}{l}
\begin{hask}{Arvore,Folha,Nodo}{White}
data Arvore a = Folha | Nodo a [Arvore a]
\end{hask}
\end{tabular}
\end{center}

Na defini��o acima, o construtor \inh{Folha} constr�i uma �rvore
vazia, ou folha, e o construtor \inh{Nodo} constr�i uma �rvore
contendo um valor de tipo \inh{a} e um lista de sub�rvores.

Outra poss�vel defini��o considera que a informa��o � armazenada nas
folhas, em vez de nos nodos internos:

\begin{center}
\begin{tabular}{l}
\begin{hask}{Avore,Folha,Nodo}{White}
data Arvore' a = Folha' a | Nodo' [Arvore' a]
\end{hask}
\end{tabular}
\end{center}

Uma �rvore com exatamente duas sub-�rvores (possivelmente vazias) �
chamada de �rvore bin�ria e pode ser definida como:

\begin{center}
\begin{tabular}{l}
\begin{hask}{ArvB,FolhaB,NodoB}{White}
 data ArvB a = FolhaB | NodoB a (ArvB a) (ArvB a)
\end{hask}
\end{tabular}
\end{center}

Em linguagens que prov�em suporte ao uso de ponteiros, mas n�o �
defini��o e manipula��o direta de tipos recursivos, a representa��o de
�rvores bin�rias pode ser feita com o uso de ponteiro como mostra o
exemplo a seguir:

\begin{center}
\begin{tabular}{l}
\begin{alg}{ArvoreBinariaDeInteiros}{White}
struct ArvoreBinariaDeInteiros
     int elem
     struct ArvoreBinariaDeInteiros *esq, dir 
\end{alg}
\end{tabular}
\end{center}

Os campos \ina{esq} e \ina{dir} de um nodo s�o ponteiros, possivelmente nulos,
para sub-�rvores.

Para �rvores n�o bin�rias, pode ser usada a seguinte representa��o, que
podemos chamar de {\em representa��o com primog�nito-irm�o-e-pai}:

\begin{center}
\begin{tabular}{l}
\begin{alg}{ArvoreDeInteiros}{White}
struct ArvoreDeInteiros
   int elem;
   struct ArvoreDeInteiros *primogenito, irmao, pai 
\end{alg}
\end{tabular}
\end{center}

%A representa��o da �rvore:

%xxxx

%� mostrada abaixo:

%yyyyy


%\chapter{An�lise da efici�ncia de algoritmos}
\label{analise-eficiencia-de-algoritmos}

Esse cap�tulo apresenta um roteiro para an�lise da efici�ncia de
algoritmos e apresenta exemplos simples de problemas e de solu��es
usando esse roteiro.

Al�m da efici�ncia, algoritmos podem ser analisados quanto a
facilidade de mostrar ou provar corre��o, simplicidade e
generalidade. 

Ao contr�rio da an�lise da efici�ncia, simplicidade e facilidade de
mostrar corre��o s�o crit�rios bastante subjetivos. � bastante dif�cil
estabelecer m�tricas para tais crit�rios. Generalidade, por sua vez,
pode ser medida pelo tamanho do dom�nio da entrada do problema
resolvido, mas h� situa��es em que o desenvolvimento de um algoritmo
mais geral � desnecess�rio (pouco vantajoso) ou dif�cil, e tal
dificuldade ou necessidade � dif�cil de ser medida precisamente.

O projeto de algoritmos envolve revis�o e busca de melhorias, com o
qual programadores devem se envolver. 

Em geral, o projeto de algortimos envolve a ado��o de solu��es que
favorecem um aspecto em detrimento de outro, e um aspecto que costuma
ser bastante influente � o tempo dispon�vel para desenvolvimento do
programa. O desenvolvimento de algoritmos {\em �timos\/} � uma quest�o
relativa ao {\em problema\/} que est� sendo resolvido e, mesmo
restringindo ao aspecto de efici�ncia, para muitos problemas saber
dizer qual � o algoritmo �timo � dif�cil e muitas vezes n�o tem uma
resposta conhecida. Vamos falar mais sobre esse assunto na se��o
{P-vs-NP}.

% Anany Levitin cita \ref{Anany-Levitin-analysis-and-design-of-algs} a
% seguinte observa��o de Saint-Exup�ry (que ele tirou de cita��o feita
% por Jon Bentley): ``Um projetista sabe que chegou � perfei��o n�o
% quando n�o h� mais nada a incluir, mas quando n�o h� mais nada a
% remover''.

A prova de corre��o de programas � uma �rea da ci�ncia da computa��o
que est� em franca evolu��o, atualmente. O desenvolvimento de teorias
de tipos \cite{Sorensen98lectureson}, baseadas nos chamados ``tipos
dependentes'' \cite{Bove:2009:DTW,Nederpelt-Geuvers-2014}, tem
evolu�do bastante. Esse desenvolvimento tem estimulado trabalhos com
os chamados ``assistentes de prova''
\cite{Geuvers2009:Proof-assistants}. Esses programas e linguagens, no
entanto, ainda requerem bastante treinamento e parecem ainda estar em
processo de evolu��o, antes que possam ser mais amplamente
usados. Atualmente, a corre��o da vasta maioria dos programas usados
na pr�tica n�o � demonstrada, mas sujeita a testes. Provas de corre��o
e t�cnicas de teste de programas n�o fazem parte do escopo deste
livro; no entanto, vamos usar provas de indu��o e defini��o de
invariantes para mostrar informalmente a corre��o de programas.

Como usualmente, n�o � feita neste livro nenhuma {\em valida��o dos
  dados de entrada}, isto �, n�o � verificado que os dados de entrada
realmente est�o dentro dos limites estabelecidos no enunciado de um
problema. Em programas usados na pr�tica, essa verifica��o deve ser
inclu�da (o enunciado do problema n�o deve estabelecer tais limites
para os dados de entrada), mas em geral essa valida��o n�o envolve
nenhum aspecto mais relevante para a tarefa de programa��o (apenas
inclus�o de testes para emiss�o de mensagens de erro no caso em que os
dados de entrada n�o satisfa�am a esses testes).

As se��es a seguir apresentam alguns exemplos de problemas para os
quais h� vamos analisar a efici�ncia de um algoritmo simples que os
resolve. O roteiro para an�lise da efici�ncia � o seguinte:

\begin{enumerate}

\item Determinar vari�vel ($n$) que representa o tamanho dos dados de
  entrada.

\item Identificar opera��es que v�o determinar a varia��o na
  efici�ncia do programa durante a execu��o.

\item Expresse o n�mero de vezes que a opera��o � executada em fun��o
  de $n$, chamada de express�o-determinante-da-efici�ncia.

\item Resolva ou simplifique a express�o-determinante-da-efici�ncia.

\end{enumerate}

No caso de um programa recursivo, a
express�o-determinante-da-efici�ncia � em geral uma {\em rela��o de
  recorr�ncia}. Uma rela��o de recorr�ncia � uma defini��o recursiva
para a qual em geral existe uma solu��o n�o recursiva que a
simplifique (especificando a mesma rela��o). No caso de um programa
n�o recursivo, a express�o-determinante-da-efici�ncia � em geral um
somat�rio, que em geral tamb�m pode ser simplificado.

As se��es seguintes apresentam exemplo de problemas simples e suas
solu��es, para os quais a efici�ncia � analisada usando o roteiro
acima.

\section{N�mero de D�gitos}
\label{numero-de-digitos}

O problema � determinar o n�mero de d�gitos de um n�mero em uma dada
base usada para representa��o desse n�mero. O n�mero e a base s�o
dados de entrada.

\subsection{Vers�o funcional}

\newcommand{\numDigs}{{\it numDigs\/}}

A vers�o funcional � apresentada em Haskell a seguir:

\progb{
\numDigs\ $x$ $b$\\
  \hspace*{.2cm} | $x$ < $b$   = 1\\
  \hspace*{.2cm} | \otherwise\ = 1 + \numDigs\ ($x$ `\ddiv` $b$)
}

A vari�vel que representa o tamanho dos dados de entrada � igual a
$n$.  A varia��o do tempo de execu��o $T(n)$ � dada por (considerando
como $k$ uma constante igual ao tempo gasto pela opera��o de somar 1 a
um valor qualquer mais o tempo gasto pela opera��o de comparar o
argumento $x$ com $b$):

 \[ \begin{array}{lll}
       T(n) & = 0                 & \text{ se } n < b\\
       T(n) & = T(n `\ddiv` b) + k & \text{ caso contr�rio}
    \end{array}
 \]
Vamos considerar que $n$ � uma pot�ncia de $b$ --- isto �, $n = b^i$,
para algum $i\geq 0$. Essa considera��o � baseada na regra .... \ldots

Para $i\geq b$, obtemos:  
 \[ \begin{array}{ll}
       T(b^i) & = T(b^{i-1}) + k \\
              & = T(b^{i-2}) + (2 \times k) \\
              & \ldots
    \end{array}
 \]
Para $n=b^i$, obtemos $T(b^i) = T(b^0) + (i\times k) = i\times k$.
Portanto, $T(n) = log_b (i\times k)$ e portanto $T(n) \asymp lg n$.

\subsection{Vers�o imperativa}

\newcommand{\numD}{{\it numD\/}}

A vers�o imperativa � similar, usando um comando de repeti��o em vez
de recurs�o:

\progb{
\numDigs\ ($n$,$b$) \{ \\
  \hspace*{.2cm} \numD\ = 0;\\
  \hspace*{.2cm}   \while\ ($n$ > $b$) \\
       \hspace*{1cm} \numD\ = \numD\ + 1\\
       \hspace*{1cm} $n$ = $n$ / $b$\\
\}
}
A express�o-determinante-da-efici�ncia � igual a $m \times \Theta(1)$,
onde $m$ � o n�mero de vezes que o comando de repeti��o � executado e
$\Theta(1)$ expressa o tempo gasto nos comandos internos ao comando de
repeti��o. Como a vari�vel $n$ recebe, a cada repeti��o, o valor do
quociente da divis�o do valor de $n$ (anterior � atribui��o) por $b$,
obtemos: $T(n) \asymp m \asymp log_b n \asymp lg n$. 

Note que $T(n) \asymp lg n$ para qualquer base $b$.

Note tamb�m que $T(n)$ (e o n�mero de repeti��es no \while) aumenta
logaritmicamente com um aumento (linear) no {\em valor\/} de $n$, mas
aumenta linearmente com um aumento no n�mero de d�gitos de $n$ (o
valor de $n$ aumenta exponencialmente com um aumento no n�mero de
d�gitos de $n$).

\section{Maior Elemento}

\section{Unicidade de Elementos}

\section{Multiplica��o de Matrizes}

\section{N�meros de Fibonacci}









%% !TEX encoding = ISO-8859-1
\chapter{Algoritmos e �rvores de Pesquisa}
\label{algoritmos-de-pesquisa}

Pesquisar em computa��o significa encontrar um dado valor, chamado de
{\em chave da pesquisa\/}, dentre v�rios valores existentes. Os
valores existentes podem estar representados de v�rias formas, mas
vamos tratar neste livro apenas de listas e �rvores.  Mesmo nos
restringindo apenas a essas formas de representa��o de valores,
existem v�rios algoritmos de pesquisa. 

Na se��o \ref{sec:pesquisa-em-lista} apresentamos um algoritmo simples
de {\em pesquisa sequencial\/} em listas (incluindo representa��o com
arranjos). Duas varia��es simples dessa pesquisa sequencial s�o
apresentadas nos exerc�cios resolvidos. A primeira � baseada em
pesquisa em lista ordenada, que termina a pesquisa sequencial quando a
chave da pesquisa � encontrada ou quando se torna maior do que um
elemento da lista (supondo ordem crescente dos valores na lista). A
segunda usa o que � chamado de {\em sentinela} --- um elemento
adicionado ao extremo (tipicamente, de arranjo), para evitar teste
para verificar chegada a esse extremo (por isso, � usada somente
quando o n�mero de elementos que pode ser armazenado � limitado, como
ocorre no caso de arranjos).

A se��o \ref{sec:pesquisa-binaria} apresenta o eficiente algoritmo de
pesquisa em um arranjo ordenado chamada de {\em pesquisa bin�ria}.  A
se��o \ref{sec:arvore-binaria-de-pesquisa} apresenta algoritmos de
pesquisa em �rvore bin�ria, incluindo varia��es do algoritmo b�sico
(em �rvores n�o balanceadas) para diferentes formas de obter
balanceamento da �rvore na qual a pesquisa � feita, com o objetivo de
aumentar a efici�ncia da pesquisa.

% !TEX encoding = ISO-8859-1
\subsection{Pesquisa binária}
\label{pesquisa-binaria}

Pesquisa binária é apresentada a seguir usando árvores de pesquisa
binária, em Haskell, e arranjos ordenados, em pseudo-código
imperativo.

Uma árvore de pesquisa binária é uma árvore binária --- isto é, um
árvore com duas sub-árvores (possivelmente vazias), digamos, à
esquerda e à direita --- com os elementos contidos nos nodos e tal
que: todo elemento contido em um nodo é maior que os elementos
contidos na sub-árvore à esquerda e menor que os elementos contidos na
sub-árvore à direita.

Dois exemplos de árvores binárias de pesquisa com os elementos de 1 a
7 são mostradas abaixo. 

  4                    4
 / \                  / \
1   5                2   6
 \   \              / \  /\
 3    6            1  3 5  7 
/      \ 
2       7

A propriedade fundamental de uma árvore binária de pesquisa é o acesso
eficiente (como vamos ver, logaritmo) a um elemento.

\subsubsection{Versão funcional}






% !TEX encoding = ISO-8859-1
\section{�rvore bin�ria de pesquisa}
\label{sec:arvore-binaria-de-pesquisa}

Uma �rvore bin�ria de pesquisa � uma �rvore bin�ria --- isto �, um
�rvore com duas sub-�rvores (possivelmente vazias), digamos, �
esquerda e � direita --- com os elementos contidos nos nodos, um
elemento por nodo, e tal que: todo elemento contido em um nodo � maior
que os elementos contidos na sub-�rvore � esquerda deste nodo e menor
que os elementos contidos na sub-�rvore � direita do nodo.

Dois exemplos de �rvores bin�rias de pesquisa com os elementos de 1 a
7 s�o mostradas abaixo. 

\begin{verbatim}
      4                    4
     / \                  / \
    1   5                2   6
     \   \              / \  /\
     3    6            1  3 5  7 
    /      \ 
    2       7
\end{verbatim}

A propriedade fundamental de uma �rvore bin�ria de pesquisa � o acesso
eficiente a um elemento (veja abaixo coment�rio sobre a complexidade
logar�tmica da pesquisa por um elemento no caso m�dio).

Uma �rvore bin�ria de pesquisa � uma importante estruturas de dados,
gerando implementa��es simples de pesquisa, inser��o e remo��o. Em
computa��o, um {\em dicion�rio\/} � um tipo abstrato que define tais
opera��es. As se��es seguintes apresentam vers�es recursivas e
iterativas de pesquisa, inser��o e remo��o, usando �rvores bin�rias de
pesquisa.

\subsection{Vers�o funcional}

Considere a defini��o de �rvore bin�ria com elementos nos nodos
internos da �rvore apresentada na se��o \ref{sec:arvores}:

\begin{center}
\begin{tabular}{l}
\begin{hask}{ArvB,FolhaB,NodoB}{White}
 data ArvB a = FolhaB | NodoB a (ArvB a) (ArvB a)
\end{hask}
\end{tabular}
\end{center}

\subsubsection{Pesquisa}

A pesquisa por um elemento em uma �rbore de pesquisa bin�ria � um
algoritmo simples, mostrado a seguir:

\begin{center}
\begin{tabular}{l}
\begin{hask}{pesq}{\decremento}
pesq v FolhaB          = False
pesq v (NodoB v' t t') = case compare v v' of
                           LT -> pesq v t
                           GT -> pesq v t'
                           _  -> True
\end{hask}
\end{tabular}
\end{center}

A complexidade de tempo de \inh{pesq} � determinada a seguir: 

\begin{enumerate}

  \item Vamos considerar que o tamanho da entrada � dado pelo n�mero
    de elementos $n$ na �rvore.

  \item A opera��es relevantes s�o a compara��o (uso de \inh{compare})
    e o casamento de padr�o.

  \item \label{complexidade-da-pesquisa-binaria-no-pior-caso} A
    express�o-determinante-da-efici�ncia �, no pior caso, igual a
    $T(n-1) + k$, onde $k$ � uma constante que expressa o tempo de
    execu��o da compara��o. No pior caso, o elemento est� em
    sub-�rvore que cont�m $n-1$ nodos: todos os nodos menos o nodo
    corrente. Neste caso, a �rvore est� desbalanceada. Isso ocorre,
    por exemplo, para as �rvores mostradas a seguir, constru�das com
    inser��es sucessivas de elementos em uma lista crescente e
    decrescente, respectivamente.

    \begin{verbatim}
         v0                             v0
          \                             /
          v1                           v1
            \                         /
            ...                     ...
              \                     /
              vn-1               vn-1
                \                 /
                 vn              vn
    \end{verbatim}

  \end{enumerate}

Assim, a rela��o de recorr�ncia neste caso �:

   \[ T(n) = T(n-1) + k \]

A solu��o dessa rela��o de recorr�ncia foi apresentada na se��o
\ref{sec:maior-elemento}:

       \[ T(n) \asymp n \]

Note que o algoritmo de pesquisa em uma �rvore bin�ria de pesquisa tem
complexidade $O(p)$ no pior caso, onde $p$ � a profundidade da �rvore
(n�mero de arestas do maior caminho existente entre a raiz e uma
folha). 

Em uma �rvore balanceada, o n�mero de elementos $n$ � igual a $2^p$,
ou seja, em uma �rvore balanceada, a pesquisa tem complexidade 

  \[ T(n) \asymp lg n \]
Em uma �rvore totalmente desbalanceada, temos no entanto $p = n-1$.

\subsubsection{Inser��o}

O algoritmo para inser��o de elementos em uma �rvore bin�ria de
pesquisa � semelhante ao algoritmo de pesquisa:

\begin{center}
\begin{tabular}{l}
\begin{hask}{ins}{\decremento}
ins v FolhaB             = NodoB v FolhaB FolhaB
ins v t@(NodoB v' t1 t2) = case compare v v' of
                             LT -> ins v t1
                             GT -> ins v t2
                             _  -> t
\end{hask}
\end{tabular}
\end{center}

O tempo de complexidade do algoritmo de inser��o � o mesmo do de
pesquisa: $lg n$ no caso de uma �rvore balanceada, mas linear no pior
caso de uma �rvore desbalanceada.

\subsubsection{Remo��o}

Para remo��o de um nodo $n$, raiz de uma �rvore $t$, o algoritmo de
remo��o de elemento essencialmente subsititui $n$ pelo menor elemento
da sub-�rvore direita de $t$ (e remove este menor elemento dessa
sub-�rvore). O algoritmo em Haskell � apresentado a seguir.

\begin{center}
\begin{tabular}{l}
\begin{hask}{remov}{\decremento}
remov:: Ord a => a -> ArvB a -> ArvB a
remov _ FolhaB = FolhaB
remov a (NodoB b l r) = case compare a b of 
                          LT -> NodoB b (remov a l) r
                          GT -> NodoB b l (remov a r)
                          _  -> junta l r

junta:: Ord a => ArvB a -> ArvB a -> ArvB a
-- junta l r cria �rvore com o menor elemento m de r como raiz e remove m de r, 
-- se m existir; sen�o (r � vazia) retorna l.
junta l r = case min r of 
              Nothing     -> l
              Just (m,r') -> Node m l r'

min:: Ord a => ArvB a -> Maybe (a,ArvB a)
-- min t retorna Just (m,r) se o menor elemento m de t existir
-- (onde r � a sub-�rvore direita de t); sen�o Nothing.
min FolhaB             = Nothing
min (NodoB a FolhaB r) = Just (a, r)
min (NodoB a l      r) = Just (m,r)
  where Just (m,_)     = min l
\end{hask}
\end{tabular}
\end{center}

O tempo de complexidade do algoritmo de remo��o de elementos � tamb�m
o mesmo do de pesquisa: $lg n$ no caso de uma �rvore balanceada, mas
linear no pior caso de uma �rvore desbalanceada.

\subsection{Vers�o imperativa}

Vamos usar a defini��o de �rvore bin�ria apresentada na se��o
\ref{sec:arvores}:

\begin{center}
\begin{tabular}{l}
\begin{alg}{ArvoreBinariaDeInteiros}{White}
struct ArvoreBinariaDeInteiros
     int elem
     struct ArvoreBinariaDeInteiros *esq, dir 
\end{alg}
\end{tabular}
\end{center}

\subsubsection{Pesquisa}

A pesquisa iterativa por um valor em uma �rvore bin�ria de pesquisa
com valores inteiros � apresentada a seguir:

\begin{center}
\begin{tabular}{l}
\begin{alg}{pesq}{\decremento}
pesq(k, arvBin) 
    nodoCorrente = arvBin
    while nocoCorrente != NULL
        v = nocoCorrente->elem
        if v > k 
           nodoCorrente = nodoCorrente->esq else
        if v < k
            nodoCorrente = nodoCorrente->dir
        else return nodoCorrente
    return NULL
\end{alg}
\end{tabular}
\end{center}

A complexidade � a mesma da vers�o recursiva: $lg n$ no caso de uma
�rvore balanceada, mas linear no pior caso de uma �rvore
desbalanceada.

\subsubsection{Inser��o}

A inser��o de elemento na vers�o imperativa sup�e que o elemento a ser
inserido � um novo elemento (n�o ocorre na �rvore). � usado comando de
repeti��o em vez de recurs�o, e comando de atribui��o � usado para
modificar a �rvore na qual o elemento est� sendo inserido, em vez de
criar nova �rvore.

Um novo nodo � criado com chamada a \ina{novoNodo}, que atribui
refer�ncias nulas aos campos \ina{esq} e \ina{dir} e atribui o
argumento ao campo \ina{elem}.

\begin{center}
\begin{tabular}{l}
\begin{alg}{ins}{\decremento}
ins (int k, ArvoreBinariaDeInteiros* arvBin) {
  ArvoreBinariaDeInteiros* nodoCorrente = arvBin
  ArvoreBinariaDeInteiros* prev = NULL
  int v
  while (nodoCorrente != NULL) 
    v = nodoCorrente->elem;
    prev = nodoCorrente;
    if (k < v) 
       nodoCorrente = nodoCorrente -> esq
    else 
       nodoCorrente = nodoCorrente -> dir

  if prev == NULL
    arvBin = novoNodo(k)
  else 
    if (k < prev->elem) 
      prev->esq = novoNodo(k)
    else 
      prev->dir = novoNodo(k)
\end{alg}
\end{tabular}
\end{center}

A complexidade de \ina{ins} � a mesma da vers�o recursiva: $lg n$ no
caso de uma �rvore balanceada, mas linear no pior caso de uma �rvore
desbalanceada.

A vers�o imperativa do algoritmo de remo��o de um elemento em uma
�rvore bin�ria de pesquisa � deixada como exerc�cio para o leitor
(exerc�cio
\ref{ex:remocao-de-elemento-em-arvore-binaria-de-pesquisa}).

\HRule

%Embora a complexidade da pesquisa a um elemento seja logar�tmica no
%caso m�dio, como vimos no item
%\ref{complexidade-da-pesquisa-binaria-no-pior-caso} acima, no pior
%caso a complexidade da pesquisa � a mesma da complexidade da pesquisa
%sequencial, quando a �rvore est� totalmente desbalanceada, ficando
%equivalente a uma lista.

H� muitos trabalhos de pesquisa em computa��o que procuram preservar o
balanceamento de �rvores bin�rias de pesquisa, para manter a
propriedade de complexidade logar�tmica para as opera��es de inser��o,
remo��o, pesquisa e ordena��o de valores, baseados principalmente em
t�cnicas de transforma��o: simplifica��o ou mudan�a de representa��o.

A simplifica��o transforma �rvores n�o balanceadas em �rvores
balanceadas (segundo alguma crit�rio de balanceamento) sem adicionar
nenhuma informa��o adicional a nodos da �rvore. A se��o
\ref{sec:arvore-AVL} aborda {\em �rvores AVL\/}, nas quais a diferen�a
entre a altura das sub-�rvores de qualquer nodo n�o deve ser maior que
1. A altura de um nodo da �rvore � o n�mero de nodos do maior caminho
entre esse nodo e uma folha. A altura de uma �rvore � altura do nodo
raiz da �rvore.

Outro estrutura de dados que prov� balanceamento de �rvores de
pesquisa baseada na t�cnica de transforma��o por simplifica��o � a que
vamos chamar de {\em �rvores MovPraRaiz\/} (em ingl�s, {\em splay
  trees\/}), baseada na ideia de mover para raiz um elemento em
opera��es de inser��o, remo��o ou pesquisa.  �rvores MovPraRaiz s�o
abordadas no exerc�cio resolvido \ref{ex:arvores-MovPraRaiz}.

Exemplos de balanceamento via mudan�a de representa��o ocorrem com os
seguintes tipos de �rvore: {\em �rvore bicolor\/} (tamb�m chamadas de
vermelha-e-preta ou rubro-negra, em ingl�s {\em red-black tree\/}) e
{\em �rvore B\/} (em ingl�s, {\em B-tree\/}), abordadas
respectivamente nas se��es \ref{sec:arvore-bicolor} e
\ref{sec:arvore-B}.


% !TEX encoding = ISO-8859-1
\section{�rvore AVL}
\label{sec:arvore-AVL}

O nome AVL � proveniente das iniciais dos sobrenomes dos dois
pesquisadores russos G.~M.~\underline{A}delson-\underline{V}elsky e
E.~M.~\underline{L}andis, que foram os primeiros a definir e realizar
trabalhos com esse tipo de �rvore.

Seja $n$ um n� de uma �rvore bin�ria, $ad_n$ e $ae_n$ as alturas da
sub-�rvore esquerda e direita de $n$, respectivamente, e seja $k_n =
ae_n - ad_n $ o {\em fator de balanceamento\/} do nodo (i.e.~o fator
de balanceamento � igual ao valor da diferen�a entre as alturas de
suas sub-�rvores).

Uma �rvore AVL � uma �rvore de pesquisa bin�ria na qual $\delta_n$ �
igual a 0 ou 1 ou -1, para todo nodo $n$.

Por exemplo, a �vore bin�ria de pesquisa abaixo � esquerda � uma
�rvore AVL, enquanto a da direita n�o �. 

\begin{verbatim}
      5                 5
     / \               /
    2   6             2 
   /                 / 
  1                 1   
\end{verbatim}

O algoritmo de pesquisa em uma �rvore AVL � o mesmo do algoritmo de
pesquisa em uma �rvore bin�ria de pesquisa. 

Os algoritmos de inser��o e remo��o s�o apresentados nas subse��es
seguintes.

\subsection{Inser��o}
\label{sec:insercao-em-arvores-AVL-versao-func}

Ap�s inser��o de nodo em uma �rvore AVL, � feita uma verifica��o do
fator de balanceamento de cada nodo que est� no caminho da raiz at� o
nodo inserido. Se a inser��o tornar o fator de balanceamento maior que
1 ou menor que -1, a sub-�rvore com raiz nesse nodo � rebalanceada,
por meio de uma {\em rota��o}, para que a condi��o-AVL volte a ser
satisfeita. Como a inser��o de elemento em uma �rvore pode aumentar a
altura da �rvore em no m�ximo 1, o fator de balanceamento deve ser,
logo ap�s a inser��o de nodo na sub-�rvore esquerda e antes do
rebalanceamento, no m�ximo igual a 2; e, logo ap�s a inser��o de nodo
na sub-�rvore direita e antes do rebalanceamento, no m�nimo igual a
-2.

Quando o fator de balanceamento � igual a 2, existem duas
possibilidades (outras duas possibilidades, que existem quando o fator
de balanceamento � igual a -2, s�o an�logas). 

\newcommand{\altura}{{\it altura\/}}

A primeira, mostrada no caso \verb+sobE+ abaixo, temos 
  $\text{\altura\ (\verb+ee+) = \altura\ (\verb+d+)}$.
Note que: 
  \begin{enumerate}
    \item se
      $\altura\ (\verb+ee+) < \altura\ (\verb+d+)$ ent�o a
      �rvore continuaria sendo uma �rvore AVL ap�s a inser��o de um
      nodo em \verb+ee+;
    \item se $\altura\ (\verb+ee+) > \altura\ (\verb+d+)$ ent�o a
      �rvore j� n�o seria uma �rvore AVL antes da inser��o de um nodo
      em \verb+ee+.
  \end{enumerate}
   
\begin{verbatim}
       Caso sobE      Caso sobED
           v              v
         /   \          /   \   
        ve    d        ve    d
       /  \           /  \                    
      ee  ed         ee   ved
                         / \
                       ede  edd                   

�rvore depois de rotacionada:

    Caso sobE          Caso sobED
     ve                   ved
    /  \                /     \ 
   ee   v             ve       v
       / \           /  \     / \ 
      ed  d         ee  ede edd  d

\end{verbatim}

O caso \verb+sobED+ pode ser expresso como \verb+sobE+ (aplicado ao
nodo com raiz \verb+ved+) seguido de \verb+sobD+ (aplicado ao mesmo
nodo).

%� importante notar que apenas um fator de balanceamento n�o �
%suficiente para determinar se uma �rvore AVL necessita de rota��o ap�s
%uma inser��o. Por exemplo, considere as duas �rvores AVL a seguir:
%
%\begin{verbatim}
%      7                  7
%     /  \               /
%    3    8             3 
%   / \    \ 
%  2   5    9
% /     \ 
%1       6
%\end{verbatim}
%
%O fator de balanceamento da �rvore com raiz \verb+7+ � igual a 1, nas
%duas �rvores acima, antes da inser��o.  No entanto, a inser��o de
%\verb+4+ n�o quebra a condi��o de a �rvore � esquerda ser AVL, ao
%contr�rio do que ocorre no caso da �rvore � direita; ap�s a inser��o,
%e antes da rota��o, que deve ser feita apenas na �rvore � direita,
%temos:
%
%\begin{verbatim}
%     AVL              N�o AVL 
%      7                  7
%     / \                /
%    3   8              3 
%   / \   \             \ 
%  2   5   9             4
% /   / \ 
%1   4   6
%\end{verbatim}

A tarefa de determinar, usando apenas o pr�prio fator de
balanceamento, a varia��o do fator de balanceamento ap�s uma inser��o,
e demonstrar que tal varia��o � verificada em todos os casos, �
deixada para trabalho futuro. N�o encontramos na literatura textos que
abordam o assunto de forma clara e precisa.

O armazenamento da altura em cada nodo evita ter que calcular a altura
de cada nodo que est� no caminho da �rvore at� o nodo inserido, o que
seria desnecessariamente ineficiente. 

\subsubsection{Vers�o funcional}

A inser��o de elemento em �rvore AVL � feita em Haskell como a seguir:

\begin{center}
\begin{tabular}{l}
\begin{hask}{ins}{\decremento}
module AVL (ArvoreAVL, arvVazia, ins) where

type Altura      = Integer
data ArvoreAVL a = Vazia | Nodo a Altura (ArvoreAVL a) (ArvoreAVL a)

arvVazia = Vazia

ins:: (Show a, Ord a) => a -> ArvoreAVL a -> ArvoreAVL a
ins k Vazia              = Nodo k 1 Vazia Vazia
ins k arv@(Nodo v _ e d) = 
 case compare k v of 
  LT -> let e1@(Nodo v1 a1 _ _) = ins k e 
            ad = altura d
         in if a1 - ad == 2 -- condi��o AVL precisa ser restaurada
            then if k < v1  
                 then sobE  (Nodo v undefined e1 d)  
                 else sobED (Nodo v undefined e1 d)
            else Nodo v (max a1 ad + 1) e1 d
  GT -> let d1@(Nodo v1 a1 _ _) = ins k d 
            ae = altura e
         in if a1 - ae == 2 -- condi��o AVL precisa ser restaurada 
            then 
               if k > v1
               then sobD  (Nodo v undefined e d1) 
               else sobDE (Nodo v undefined e d1) 
            else Nodo v (max ae a1 + 1) e d1
  _ -> arv

sobE :: ArvoreAVL a -> ArvoreAVL a
sobE (Nodo v _ (Nodo ve _ ee ed) d) = Nodo ve a ee (Nodo v ad ed d) 
  where ad = max (altura ed) (altura d) + 1
        a  = max (altura ee) ad         + 1

sobD :: ArvoreAVL a -> ArvoreAVL a
sobD (Nodo v _ e (Nodo vd _ de dd)) = Nodo vd a (Nodo v ae e de) dd
  where ae = max (altura e) (altura de) + 1
        a  = max ae         (altura dd) + 1

sobED :: ArvoreAVL a -> ArvoreAVL a
sobED (Nodo v _ (Nodo ve _ ee (Nodo ved _ ede edd)) d) = 
       Nodo ved a (Nodo ve ae ee ede) (Nodo v ad edd d)
  where a  = max ae           ad           + 1
        ae = max (altura ee ) (altura ede) + 1
        ad = max (altura edd) (altura d  ) + 1

sobDE :: ArvoreAVL a -> ArvoreAVL a
sobDE (Nodo v _ e (Nodo vd _ (Nodo vde _ dee ded) dd)) = 
       Nodo vde a (Nodo v ae e dee) (Nodo vd ad ded dd)
  where a  = max ae           ad           + 1
        ad = max (altura ded) (altura dd ) + 1
        ae = max (altura e  ) (altura dee) + 1 

altura (Nodo _ a _ _) = a
altura Vazia          = 0
\end{hask}
\end{tabular}
\end{center}

No pior caso, temos $T(h) = T(h-1) + k$, onde $h$ � a altura da �rvore
e $k$ � o tempo de execu��o referente aos c�lculos de i) altura de uma
sub-�rvore, ii) condi��o AVL e iii) rota��o (uma das fun��es
\inh{sobE}, \inh{sobED}, \inh{sobD}, \inh{sobDE}). Todos os tempos de
i) a iii) t�m complexidade $O(1)$. Logo (cf.~se��o
\ref{sec:maior-elemento}): $T(h) \asymp h$, ou seja, considerando que
$h \asymp lg n$ em uma �rvore balanceada (onde $n$ � o n�mero de
elementos da �rvore):

  \[ T(n) \asymp lg n \]

\subsubsection{Vers�o imperativa}

A vers�o imperativa, mostrada abaixo, usa:
  
  \begin{itemize}

    \item fun��o \ina{novoNodo} e fun��o \ina{malloc}, como em \C:
      \ina{malloc} i) aloca �rea de mem�ria para conter registro
      \ina{AVL} --- o tamanho da �rea a ser alocada � passada para
      \ina{malloc}, sendo o tamanho calculado pela fun��o \ina{sizeof}
      ---, e ii) retorna apontador para �rea alocada;

    \item express�es condicionais, como em \C\ (introduzida na se��o
      \ref{pesquisa-sequencial-em-arranjo-versao-imp});

    \item em \C, � necess�rio definir: \ina{typedef struct AVL AVL;}
      para poder usar apenas \ina{AVL} em vez de \ina{struct AVL};
      al�m disso, o uso de \ina{struct AVL} em vez de apenas \ina{AVL}
      em campo de \ina{AVL} � devido a como � definido (requerido) na
      linguagem \C.

 \end{itemize}

\begin{center}
\begin{tabular}{l}
\begin{alg}{ins}{\decremento}
struct AVL 
  int chave, altura
  struct AVL *esq, *dir

int altura (AVL *p)
  return (p == NULL ? 0 : p->altura)
 
int max (int a, int b) 
  return (a > b ? a : b)
 
AVL* novoNodo (int chave)
    AVL* nodo    = (AVL*) malloc(sizeof(AVL))
    nodo->chave  = chave
    nodo->esq    = nodo->dir = NULL
    nodo->altura = 1
    return nodo

AVL* sobE (AVL* v)
    AVL* ve    = v ->esq
    AVL* ved   = ve->dir
    ve->dir    = v
    v ->esq    = ved
    v ->altura = max (altura (v->esq ), altura (v->dir ))+1
    ve->altura = max (altura (ve->esq), v->altura       )+1
    return ve
 
AVL* sobD (AVL* v) 
    AVL* vd    = v ->dir
    AVL* vde   = vd->esq
    vd->esq    = v
    v ->dir    = vde
    v ->altura = max (altura (v->esq ), altura (v->dir ))+1
    vd->altura = max (v->altura       , altura (vd->dir))+1
    return vd

AVL* sobED (AVL* v) 
  v->esq = sobD (v->esq)  // v->esq->dir sobe (se torna v->esq)
  return sobE(v)          // v->esq      sobe (se torna v)

AVL* sobDE (AVL* v)
  v->dir = sobE (v->dir)  // v->dir->esq sobe (se torna v->dir)
  return sobD(v)          // v->dir      sobe (se torna v)

AVL* ins (AVL* nodo, int chave)
    if (nodo == NULL) return novoNodo (chave)
    if (chave < nodo->chave)
      nodo->esq    = ins (nodo->esq, chave)
      nodo->altura = max (altura (nodo->esq), altura (nodo->dir)) + 1
     if (altura (nodo->esq) - altura (nodo->dir) == 2) // condi��o AVL precisa ser restaurada
	if (chave < nodo->esq->chave) 
	  return sobE (nodo)
	else return sobED (nodo)
    else if (chave > nodo->chave) 
            nodo->dir    = ins (nodo->dir, chave)
            nodo->altura = max (altura (nodo->esq), altura (nodo->dir)) + 1
            if (altura(nodo->dir) - altura(nodo->esq) == 2)  // condi��o AVL precisa ser restaurada
	       if (chave > nodo->dir->chave)
	          return sobD (nodo)
	       else return sobDE (nodo)

    return nodo;
\end{alg}
\end{tabular}
\end{center}

A complexidade de tempo de \inh{ins} � a mesma da vers�o funcional:
$T(n) \asymp lg n$.

\subsection{Remo��o}

\subsubsection{Vers�o funcional}

\subsubsection{Vers�o imperativa}


\section{Árvore Bicolor}
\label{sec:arvore-bicolor}

Uma árvore bicolor --- ou rubro-negra, ou vermelha-e-preta --- é uma
árvore binária de pesquisa com uma informação em cada nodo que indica
sua cor --- vamos chamar um nodo de cor $A$ de nodo $A$ e um nodo de
cor $B$ de nodo $B$ (em geral, as cores $A$ e $B$ são vermelha e
preta) --- e que deve além disso satisfazer às seguintes condições
(chamadas de {\em condições-de-árvore-bicolor\/}):

\begin{enumerate}

\item As folhas e a raiz são nodos $B$.

\item Os filhos de todo nodo $A$ são nodos $B$.

\item Todo caminho de um nodo a uma folha contém o mesmo número de
  nodos $B$.

\end{enumerate}

Um exemplo de uma árvore bicolor é mostrada abaixo.

.... aqui vem figura de árvore vermelha-e-preta. ....

As duas últimas condições garantem a propriedade {\em
  balanceamento-bicolor\/}: o comprimento do caminho mais longo da
raiz a uma folha não é maior que o dobro do comprimento do menor
caminho da raiz a uma folha. Ou seja, a árvore é razoavelmente bem
balanceada em termos de altura dos seus nodos (altura = comprimento do
maior caminho do nodo até uma folha). Como o pior caso da complexidade
de tempo de operações de inserção, remoção e pesquisa são
proporcionais à altura da árvore, isso permite uma eficiência maior do
que árvores binárias de pesquisa.

Para ver porque as duas últimas condições garantem a propriedade de
balanceamento-bicolor, considere que $n_p$ seja o número de nodos $B$
de um caminho a uma folha, para uma árvore $a$.  Seja $k$ o
comprimento do menor caminho $c_m$ da raiz a uma folha (esse caminho
contém $n_b$ nodos pretos). O caminho mais longo da raiz a uma folha
$c_M$ não pode conter nodos vermelhos em níveis consecutivos da árvore
(um novo vermelho não pode ser filho de outro nodo vermelho), e o
número de nodos $B$ tem que ser igual a $n_p$. Portanto, o número de
nodos de $c_M$ tem que ser no máximo igual a $2*k$ (caso em que $c_m$
só tem nodos $B$).

Operações que modificam a árvore geralmente causam reorganização dos
nodos e cores dos nodos de modo a que as condições-de-árvore-bicolor
continuem a ser satisfeitas, o que pode ser feito de modo eficiente.

% The balancing of the tree is not perfect but it is good enough to
% allow it to guarantee searching in O(log n) time, where n is the total
% number of elements in the tree. The insertion and deletion operations,
% along with the tree rearrangement and recoloring, are also performed
% in O(log n) time.

% !TEX encoding = ISO-8859-1
\section{�rvore B}
\label{sec:arvore-B}


\section{Exerc�cios Resolvidos}

\begin{enumerate}

\item Escreva fun��o que implementa algoritmo de pesquisa em lista
  ordenada.  Ou seja, escreva fun��o que recebe um valor e uma lista,
  que pode-se supor que est� ordenada, em ordem n�o-decrescente, e
  retorne valor booleano que indica se o valor est� presente ou n�o na
  lista. A pesquisa deve considerar, por quest�o de efici�ncia, que a
  lista est� ordenada.

{\bf Solu��o}: 

Vers�o funcional:

\begin{center}
\begin{tabular}{l}
\begin{hask}{elemOrdL}{\decremento}
elemOrdL :: Ord a => a -> [a] -> Bool
elemOrdL _ [] = False
elemOrdL a (b:x) 
  | a < b     = elemOrdL a x
  | otherwise = a == b
\end{hask}
\end{tabular}
\end{center}

Vers�o imperativa:

\begin{center}
\begin{tabular}{l}
\begin{alg}{elemOrdL}{\decremento}
elemOrdL (a, l) 
   while ((l != NULL) && (l->elem < a)) l = l->r
   return (l->elem == a)
\end{alg}
\end{tabular}
\end{center}

\item \label{ex:lista-duplamente-encadeada} ....

\item \label{ex:pesquisa-em-arranjo-com-sentinela} .... Para evitar o
  teste .... pesquisa em arranjo com sentinela ....

\item \label{ex:dicionario} Em computa��o, um dicion�rio � um tipo
  abstrato que define opera��es de inser��o, remo��o e
  pesquisa. Define um tipo abstrato dicion�rio usando uma �rvore
  bin�ria de pesquisa para implementa��o das opera��es sobre o tipo
  abstrato.

  {\bf Solu��o}: ....

\item \label{ex:arvores-MovPraRaiz} Opera��es em uma �rvore MovPraRaiz
  (como inser��o, remo��o ou pesquisa) podem ser seguidas por uma
  opera��o b�sica chamada de {\em movPraRaiz\/} (em ingl�s, {\em
    splaying\/}), que reorganiza a �rvore de modo a mover um elemento
  para a raiz (em opera��es de inser��o e pesquisa, o elemento
  inserido ou pesquisado; em opera��o de remo��o, o elemento a ser
  removido � movido para a raiz antes de ser removido).  A ideia
  b�sica por tr�s disso � tornar os elementos usados mais recentemente
  pr�ximos � raiz da �rvore e a �rvore razoavelmente balanceada.

  A opera��o \ina{movPraRaiz}, que move um nodo $x$ para a raiz, pode
  realizar um dentre os seguintes passos:
 
  \begin{enumerate}

    \item passo x-raiz: se $x$ � filho da raiz,

    \item passo x-pai-de-x-no-mesmo-lado: 
          $x$ � filho do lado esquerdo e o pai de $x$ est� � esquerda da raiz, ou 
          $x$ � filho do lado direito  e o pai de $x$ est� � direita da raiz, 

    \item passo x-pai-de-x-em-lados-opostos: 
          $x$ � filho do lado esquerdo e o pai de $x$ est� � direita da raiz, ou 
          $x$ � filho do lado direito  e o pai de $x$ est� � esquerda da raiz.

  \end{enumerate}

Essas tr�s possibilidades s�o ilustradas abaixo.

...

\end{enumerate}

\section{Exerc�cios}

\begin{enumerate}

\item \label{remocao-de-elemento-em-arvore-binaria-de-pesquisa}
  Escreva um programa para remo��o de um elemento em uma �rvore
  bin�ria de pesquisa que use comandos de repeti��o em vez de fun��es
  definidas recursivamente.

\item \label{experimente-alg-ins-arv-AVL} Escreva um programa para
  percorrer e imprimir �rvores resultantes de inser��es repetidas de
  elementos em uma �rvore vazia, usando vers�o imperativa e/ou
  funcional do algoritmo de inser��o em �rvore AVL.

\item \label{experimente-alg-rem-arv-AVL} Escreva um programa para
  percorrer e imprimir �rvores resultantes de inser��es repetidas com
  remo��es de alguns elementos inseridos, em uma �rvore vazia, usando
  vers�o imperativa e/ou funcional dos algoritmos de inser��o e
  remo��o em �rvore AVL.

\end{enumerate}


%\chapter{Algoritmos de Ordena��o}
\label{algoritmos-de-ordenacao}

\section{QuickSort}
\label{sec:quickSort}


\section{MergeSort}
\label{sec:mergeSort}



%% !TEX encoding = ISO-8859-1
\chapter{Algoritmos de Dispers�o (Hash)}
\label{dispersao}

Este cap�tulo apresenta uma t�cnica simples e eficiente de
implementa��o de dicion�rios que ilustra bem o compromisso entre tempo
e espa�o no projeto de implementa��o de algoritmos e estruturas de
dados. O uso de mais espa�o pode diminuir o tempo e o uso de menos
espa�o pode aumentar o tempo de execu��o, mas h� sempre um limite
superior para a quantidade de espa�o a ser utilizada.

\index{dicion�rio} Em computa��o, um dicion�rio � um tipo abstrato que
define opera��es de inser��o, remo��o e pesquisa.

Os dados s�o usualmente organizados em registros com diversos campos,
dentre eles um ou mais campos que s�o usados como {\em chave\/} para a
pesquisa, inser��o ou remo��o.

\index{fun��o de dispers�o} \index{�ndice de dispers�o} Algoritmos de
dispers�o (em ingl�s, {\em hashing\/}) se baseiam na ideia simples de
distribuir os dados em um vetor de certo tamanho --- que vamos chamar
de tabela de dispers�o --- usando uma {\em fun��o de dispers�o\/}, que
associa um �ndice do vetor a cada chave.

Por exemplo, se a chave � um inteiro positivo ou nulo e a tabela de
dispers�o tem tamanho $m$, a fun��o de dispers�o pode ser a fun��o
$\disperse$ tal que $\disperse(v) = v \% m$, onde $\%$ � a fun��o que
calcula o resto da divis�o de n�meros inteiros.

% Se a chave for um caractere, a fun��o de dispers�o $\disperse$ pode
% ser tal que $\delta(v) = \ord(v) \% m$, onde $\ord$ � a fun��o que
% associa a cada caractere o c�digo inteiro que o
% representa. Computadores usam um c�digo para representar cada
% caractere; hoje em dia usualmente Unicode ou uma varia��o do c�digo
% Unicode, antigamente usava-se c�digo Ascii.

A distribui��o dos valores a serem pesquisados em uma tabela � um bom
exemplo do compromisso entre tempo e espa�o, fundamental no projeto de
algoritmos. Se n�o houver limita��o de mem�ria, poderia ser definido
um tamanho de vetor bastante grande de modo que um �nico �ndice fosse
associado a cada valor. Se n�o houver limita��o de tempo, o projeto
poderia n�o usar dispers�o (i.e.~considerar vetor de tamanho nulo) e
usar algoritmos sequenciais em uma lista. Na pr�tica, deve-se procurar
determinar um tamanho que n�o gaste espa�o de mem�ria de mais e nem de
menos, para a tabela de dispers�o. H� algoritmos que dinamicamente
alteram (tipicamente, dobram) o tamanho da tabela, quando o {\em fator
  de ocupa��o\/} --- n�mero de elementos existentes dividido pelo
tamanho (n�mero de �ndices) da tabela --- alcan�a um valor
pr�-estabelecido (igual ou pouco maior que 0,5).

\index{algoritmos de dispers�o!aberto} \index{algoritmos de
  dispers�o!fecheado} Quando o tamanho da tabela � menor que o n�mero
de chaves, pode haver {\em colis�es\/}: uma colis�o ocorre quando a
fun��o de dispers�o retorna o mesmo �ndice para duas ou mais
chaves. Colis�es devem ocorrer raramente se o tamanho da tabela de
dispers�o � grande e uma boa fun��o de dispers�o � usada. Mas um
mecanismo de tratamento de colis�es � necess�rio em todo algoritmo de
dispers�o (i.e.~sempre que o tamanho da tabela � menor que o n�mero de
chaves).

A dispers�o de valores se baseia no fato de que � mais eficiente
procurar um valor em um subconjunto de todos os valores, i.e.~o
subconjunto dos valores para os quais a fun��o de dispers�o fornece o
mesmo resultado. A pesquisa usando dispers�o � o m�todo conhecido mais
poderoso e o mais usado para pesquisa de dados. A maioria dos sistemas
de recupera��o de dados usados atualmente s�o baseados em dispers�o.

H� duas vers�es de tratamento de colis�es em algoritmos de dispers�o:
{\em aberto\/} e {\em fechado}. 

No tratamento aberto, tamb�m chamado de tratamento de colis�o por {\em
  encadeamento separado\/} (em ingl�s, {\em separate chaining\/}), a
tabela de dispers�o � uma tabela de apontadores para lista de
elementos para os quais houve colis�o.

No tratamento fechado, tamb�m chamado de {\em endere�amento aberto\/}
(em ingl�s, {\em open addressing\/}), todas as chaves s�o armazenadas
na pr�pria tabela (o que implica que o tamanho da tabela � maior que o
n�mero de chaves inseridas). Diferentes estrat�gias podem ser usadas
para resolu��o de conflitos, mas a mais simples --- chamada de {\em
  sondagem linear\/} (em ingl�s, {\em linear probing\/}) --- usa a
primeira posi��o, seguinte � que ocorreu a colis�o, que est� vazia
(considerando a tabela como circular, isto �, a primeira posi��o da
tabela segue a �ltima). Embora pesquisa e inser��o sejam relativamente
simples de implementar segundo a t�cnica de sondagem linear, a remo��o
de chaves � mais complicada. Em geral, � usado um s�mbolo
especialmente reservado para indicar que uma chave foi removida da
posi��o. N�o vamos abordar o tratamento fechado de colis�es.




%\pagebreak
%\thispagestyle{empty}
%\pagebreak
%\pagestyle{fancyplain}

%\bibliographystyle{plain}
%\bibliography{livro}

%\appendix
%%!TEX encoding = ISO-8859-1
\chapter{\Haskell}
\label{ap:Haskell}

A linguagem \Haskell\ \ldots

\section{Classes de tipos}
\label{sec:Classes-de-tipos}

\section{Defini��o de lista por gera��o e filtragem}
\label{sec:definicao-de-lista-por-geracao-e-filtragem}


%\chapter{Rela��es de Recorr�ncia}
\label{relacoes-de-recorrencia}



%\printindex

\end{document}
