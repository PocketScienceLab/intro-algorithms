\chapter{Algoritmos de Dispersão (Hash)}
\label{dispersao}

Este capítulo apresenta uma técnica simples e eficiente de
implementação de dicionários que ilustra bem o compromisso entre tempo
e espaço no projeto de implementação de algoritmos e estruturas de
dados. O uso de mais espaço pode diminuir o tempo e o uso de menos
espaço pode aumentar o tempo de execução, mas há sempre um limite
superior para a quantidade de espaço a ser utilizada.

\index{dicionário} Em computação, um dicionário é um tipo abstrato que
define operações de inserção, remoção e pesquisa.

Os dados são usualmente organizados em registros com diversos campos,
dentre eles um ou mais campos que são usados como {\em chave\/} para a
pesquisa, inserção ou remoção.

\index{função de dispersão} \index{índice de dispersão} Algoritmos de
dispersão (em inglês, {\em hashing\/}) se baseiam na ideia simples de
distribuir os dados em um vetor de certo tamanho --- que vamos chamar
de tabela de dispersão --- usando uma {\em função de dispersão\/}, que
associa um índice do vetor a cada chave. O valor retornado pela função
de dispersão é chamado de {\em índice de dispersão\/}.

Por exemplo, se a chave é um inteiro positivo ou nulo e a tabela de
dispersão tem tamanho $m$, a função de dispersão pode ser a função
$\disperse$ tal que $\disperse(v) = v \% m$, onde $\%$ é a função que
calcula o resto da divisão de números inteiros.

% Se a chave for um caractere, a função de dispersão $\disperse$ pode
% ser tal que $\delta(v) = \ord(v) \% m$, onde $\ord$ é a função que
% associa a cada caractere o código inteiro que o
% representa. Computadores usam um código para representar cada
% caractere; hoje em dia usualmente Unicode ou uma variação do código
% Unicode, antigamente usava-se código Ascii.

A distribuição dos valores a serem pesquisados em uma tabela é um bom
exemplo do compromisso entre tempo e espaço, fundamental no projeto de
algoritmos. Se não houver limitação de memória, poderia ser definido
um tamanho de vetor bastante grande de modo que um único índice fosse
associado a cada valor. Se não houver limitação de tempo, o projeto
poderia não usar dispersão (i.e.~considerar vetor de tamanho nulo) e
usar algoritmos sequenciais em uma lista. Na prática, deve-se procurar
determinar um tamanho que não gaste espaço de memória de mais e nem de
menos, para a tabela de dispersão. Há algoritmos que dinamicamente
alteram (tipicamente, dobram) o tamanho da tabela, quando o {\em fator
  de ocupação\/} --- número de elementos existentes dividido pelo
tamanho (número de índices) da tabela --- alcança um valor
pré-estabelecido (igual ou pouco maior que 0,5).

\index{algoritmos de dispersão!aberto} \index{algoritmos de
  dispersão!fecheado} Quando o tamanho da tabela é menor que o número
de chaves, pode haver {\em colisões\/}: uma colisão ocorre quando a
função de dispersão retorna o mesmo índice para duas ou mais
chaves. Colisões devem ocorrer raramente se o tamanho da tabela de
dispersão é grande e uma boa função de dispersão é usada. Mas um
mecanismo de tratamento de colisões é necessário em todo algoritmo de
dispersão (i.e.~sempre que o tamanho da tabela é menor que o número de
chaves).

A dispersão de valores se baseia no fato de que é mais eficiente
procurar um valor em um subconjunto de todos os valores, i.e.~o
subconjunto dos valores para os quais a função de dispersão fornece o
mesmo resultado. A pesquisa usando dispersão é o método conhecido mais
poderoso e o mais usado para pesquisa de dados. A maioria dos sistemas
de recuperação de dados usados atualmente são baseados em dispersão.

Há duas versões de tratamento de colisões em algoritmos de dispersão:
{\em aberto\/} e {\em fechado}. 

No tratamento aberto, também chamado de tratamento de colisão por {\em
  encadeamento separado\/} (em inglês, {\em separate chaining\/}), a
tabela de dispersão é uma tabela de apontadores para lista de
elementos para os quais houve colisão.

No tratamento fechado, também chamado de {\em endereçamento aberto\/}
(em inglês, {\em open addressing\/}), todas as chaves são armazenadas
na própria tabela (o que implica/requer que o tamanho da tabela seja
maior que o número de chaves inseridas). Diferentes estratégias podem
ser usadas para resolução de conflitos, mas a mais simples --- chamada
de {\em sondagem linear\/} (em inglês, {\em linear probing\/}) --- usa
a primeira posição, seguinte à que ocorreu a colisão, que está vazia
(considerando a tabela como circular, isto é, a primeira posição da
tabela segue a última). Embora pesquisa e inserção sejam relativamente
simples de implementar segundo a técnica de sondagem linear, a remoção
de chaves é mais complicada. Em geral, é usado um símbolo
especialmente reservado para indicar que uma chave foi removida da
posição. Não vamos abordar o tratamento fechado de colisões.


