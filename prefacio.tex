\chapter{\colorbox{cyan}{Pref�cio}}

Este livro prov� uma introdu��o ao estudo de algoritmos. Ele apresenta
estruturas de dados b�sicas e algoritmos de pesquisa e ordena��o, e
prov� uma introdu��o ao importante ramo da ci�ncia da computa��o que
trata do desenvolvimento e an�lise da efici�ncia de algoritmos. O
assunto da an�lise de efici�ncia � comumente chamado em computa��o de
complexidade.

Essa an�lise aborda em geral quanto tempo � gasto na execu��o de um
algoritmo em fun��o do tamanho da entrada: diz-se complexidade de
tempo do algoritmo. Al�m do tempo, pode ser analisada tamb�m a
complexidade de espa�o (quanto espa�o de mem�ria � gasto na execu��o
em fun��o do tamanho da entrada).

... nota��o ...

... funcional ...

... clareza, concis�o ...

\subsection{Conte�do e Organiza��o do Livro}

\subsection{Recursos Adicionais}

\subsection{Pr�-requisitos}

Os pr�-requsitos s�o:

\begin{enumerate}

\item Experi�ncia inicial com provas por indu��o. (??)

\end{enumerate}
