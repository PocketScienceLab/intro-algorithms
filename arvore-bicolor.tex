\section{Árvore Bicolor}
\label{sec:arvore-bicolor}

Uma árvore bicolor --- ou rubro-negra, ou vermelha-e-preta --- é uma
árvore binária de pesquisa com uma informação em cada nodo que indica
sua cor --- vamos chamar um nodo de cor $A$ de nodo $A$ e um nodo de
cor $B$ de nodo $B$ (em geral, as cores $A$ e $B$ são vermelha e
preta) --- e que deve além disso satisfazer às seguintes condições
(chamadas de {\em condições-de-árvore-bicolor\/}):

\begin{enumerate}

\item As folhas e a raiz são nodos $B$.

\item Os filhos de todo nodo $A$ são nodos $B$.

\item Todo caminho de um nodo a uma folha contém o mesmo número de
  nodos $B$.

\end{enumerate}

Um exemplo de uma árvore bicolor é mostrada abaixo.

.... aqui vem figura de árvore vermelha-e-preta. ....

As duas últimas condições garantem a propriedade {\em
  balanceamento-bicolor\/}: o comprimento do caminho mais longo da
raiz a uma folha não é maior que o dobro do comprimento do menor
caminho da raiz a uma folha. Ou seja, a árvore é razoavelmente bem
balanceada em termos de altura dos seus nodos (altura = comprimento do
maior caminho do nodo até uma folha). Como o pior caso da complexidade
de tempo de operações de inserção, remoção e pesquisa são
proporcionais à altura da árvore, isso permite uma eficiência maior do
que árvores binárias de pesquisa.

Para ver porque as duas últimas condições garantem a propriedade de
balanceamento-bicolor, considere que $n_p$ seja o número de nodos $B$
de um caminho a uma folha, para uma árvore $a$.  Seja $k$ o
comprimento do menor caminho $c_m$ da raiz a uma folha (esse caminho
contém $n_b$ nodos pretos). O caminho mais longo da raiz a uma folha
$c_M$ não pode conter nodos vermelhos em níveis consecutivos da árvore
(um novo vermelho não pode ser filho de outro nodo vermelho), e o
número de nodos $B$ tem que ser igual a $n_p$. Portanto, o número de
nodos de $c_M$ tem que ser no máximo igual a $2*k$ (caso em que $c_m$
só tem nodos $B$).

Operações que modificam a árvore geralmente causam reorganização dos
nodos e cores dos nodos de modo a que as condições-de-árvore-bicolor
continuem a ser satisfeitas, o que pode ser feito de modo eficiente.

% The balancing of the tree is not perfect but it is good enough to
% allow it to guarantee searching in O(log n) time, where n is the total
% number of elements in the tree. The insertion and deletion operations,
% along with the tree rearrangement and recoloring, are also performed
% in O(log n) time.
