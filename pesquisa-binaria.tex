\subsection{Pesquisa binária}
\label{pesquisa-binaria}

Pesquisa binária é apresentada a seguir usando árvores de pesquisa
binária, em Haskell, e arranjos ordenados, em pseudo-código
imperativo.

Uma árvore de pesquisa binária é uma árvore binária --- isto é, um
árvore com duas sub-árvores (possivelmente vazias), digamos, à
esquerda e à direita --- com os elementos contidos nos nodos e tal
que: todo elemento contido em um nodo é maior que os elementos
contidos na sub-árvore à esquerda e menor que os elementos contidos na
sub-árvore à direita.

Dois exemplos de árvores binárias de pesquisa com os elementos de 1 a
7 são mostradas abaixo. 

  4                    4
 / \                  / \
1   5                2   6
 \   \              / \  /\
 3    6            1  3 5  7 
/      \ 
2       7

A propriedade fundamental de uma árvore binária de pesquisa é o acesso
eficiente (como vamos ver, logaritmo) a um elemento.

\subsubsection{Versão funcional}





